
% Default to the notebook output style

    


% Inherit from the specified cell style.




    
\documentclass[11pt]{article}

    
    
    \usepackage[T1]{fontenc}
    % Nicer default font (+ math font) than Computer Modern for most use cases
    \usepackage{mathpazo}

    % Basic figure setup, for now with no caption control since it's done
    % automatically by Pandoc (which extracts ![](path) syntax from Markdown).
    \usepackage{graphicx}
    % We will generate all images so they have a width \maxwidth. This means
    % that they will get their normal width if they fit onto the page, but
    % are scaled down if they would overflow the margins.
    \makeatletter
    \def\maxwidth{\ifdim\Gin@nat@width>\linewidth\linewidth
    \else\Gin@nat@width\fi}
    \makeatother
    \let\Oldincludegraphics\includegraphics
    % Set max figure width to be 80% of text width, for now hardcoded.
    \renewcommand{\includegraphics}[1]{\Oldincludegraphics[width=.8\maxwidth]{#1}}
    % Ensure that by default, figures have no caption (until we provide a
    % proper Figure object with a Caption API and a way to capture that
    % in the conversion process - todo).
    \usepackage{caption}
    \DeclareCaptionLabelFormat{nolabel}{}
    \captionsetup{labelformat=nolabel}

    \usepackage{adjustbox} % Used to constrain images to a maximum size 
    \usepackage{xcolor} % Allow colors to be defined
    \usepackage{enumerate} % Needed for markdown enumerations to work
    \usepackage{geometry} % Used to adjust the document margins
    \usepackage{amsmath} % Equations
    \usepackage{amssymb} % Equations
    \usepackage{textcomp} % defines textquotesingle
    % Hack from http://tex.stackexchange.com/a/47451/13684:
    \AtBeginDocument{%
        \def\PYZsq{\textquotesingle}% Upright quotes in Pygmentized code
    }
    \usepackage{upquote} % Upright quotes for verbatim code
    \usepackage{eurosym} % defines \euro
    \usepackage[mathletters]{ucs} % Extended unicode (utf-8) support
    \usepackage[utf8x]{inputenc} % Allow utf-8 characters in the tex document
    \usepackage{fancyvrb} % verbatim replacement that allows latex
    \usepackage{grffile} % extends the file name processing of package graphics 
                         % to support a larger range 
    % The hyperref package gives us a pdf with properly built
    % internal navigation ('pdf bookmarks' for the table of contents,
    % internal cross-reference links, web links for URLs, etc.)
    \usepackage{hyperref}
    \usepackage{longtable} % longtable support required by pandoc >1.10
    \usepackage{booktabs}  % table support for pandoc > 1.12.2
    \usepackage[inline]{enumitem} % IRkernel/repr support (it uses the enumerate* environment)
    \usepackage[normalem]{ulem} % ulem is needed to support strikethroughs (\sout)
                                % normalem makes italics be italics, not underlines
    

    
    
    % Colors for the hyperref package
    \definecolor{urlcolor}{rgb}{0,.145,.698}
    \definecolor{linkcolor}{rgb}{.71,0.21,0.01}
    \definecolor{citecolor}{rgb}{.12,.54,.11}

    % ANSI colors
    \definecolor{ansi-black}{HTML}{3E424D}
    \definecolor{ansi-black-intense}{HTML}{282C36}
    \definecolor{ansi-red}{HTML}{E75C58}
    \definecolor{ansi-red-intense}{HTML}{B22B31}
    \definecolor{ansi-green}{HTML}{00A250}
    \definecolor{ansi-green-intense}{HTML}{007427}
    \definecolor{ansi-yellow}{HTML}{DDB62B}
    \definecolor{ansi-yellow-intense}{HTML}{B27D12}
    \definecolor{ansi-blue}{HTML}{208FFB}
    \definecolor{ansi-blue-intense}{HTML}{0065CA}
    \definecolor{ansi-magenta}{HTML}{D160C4}
    \definecolor{ansi-magenta-intense}{HTML}{A03196}
    \definecolor{ansi-cyan}{HTML}{60C6C8}
    \definecolor{ansi-cyan-intense}{HTML}{258F8F}
    \definecolor{ansi-white}{HTML}{C5C1B4}
    \definecolor{ansi-white-intense}{HTML}{A1A6B2}

    % commands and environments needed by pandoc snippets
    % extracted from the output of `pandoc -s`
    \providecommand{\tightlist}{%
      \setlength{\itemsep}{0pt}\setlength{\parskip}{0pt}}
    \DefineVerbatimEnvironment{Highlighting}{Verbatim}{commandchars=\\\{\}}
    % Add ',fontsize=\small' for more characters per line
    \newenvironment{Shaded}{}{}
    \newcommand{\KeywordTok}[1]{\textcolor[rgb]{0.00,0.44,0.13}{\textbf{{#1}}}}
    \newcommand{\DataTypeTok}[1]{\textcolor[rgb]{0.56,0.13,0.00}{{#1}}}
    \newcommand{\DecValTok}[1]{\textcolor[rgb]{0.25,0.63,0.44}{{#1}}}
    \newcommand{\BaseNTok}[1]{\textcolor[rgb]{0.25,0.63,0.44}{{#1}}}
    \newcommand{\FloatTok}[1]{\textcolor[rgb]{0.25,0.63,0.44}{{#1}}}
    \newcommand{\CharTok}[1]{\textcolor[rgb]{0.25,0.44,0.63}{{#1}}}
    \newcommand{\StringTok}[1]{\textcolor[rgb]{0.25,0.44,0.63}{{#1}}}
    \newcommand{\CommentTok}[1]{\textcolor[rgb]{0.38,0.63,0.69}{\textit{{#1}}}}
    \newcommand{\OtherTok}[1]{\textcolor[rgb]{0.00,0.44,0.13}{{#1}}}
    \newcommand{\AlertTok}[1]{\textcolor[rgb]{1.00,0.00,0.00}{\textbf{{#1}}}}
    \newcommand{\FunctionTok}[1]{\textcolor[rgb]{0.02,0.16,0.49}{{#1}}}
    \newcommand{\RegionMarkerTok}[1]{{#1}}
    \newcommand{\ErrorTok}[1]{\textcolor[rgb]{1.00,0.00,0.00}{\textbf{{#1}}}}
    \newcommand{\NormalTok}[1]{{#1}}
    
    % Additional commands for more recent versions of Pandoc
    \newcommand{\ConstantTok}[1]{\textcolor[rgb]{0.53,0.00,0.00}{{#1}}}
    \newcommand{\SpecialCharTok}[1]{\textcolor[rgb]{0.25,0.44,0.63}{{#1}}}
    \newcommand{\VerbatimStringTok}[1]{\textcolor[rgb]{0.25,0.44,0.63}{{#1}}}
    \newcommand{\SpecialStringTok}[1]{\textcolor[rgb]{0.73,0.40,0.53}{{#1}}}
    \newcommand{\ImportTok}[1]{{#1}}
    \newcommand{\DocumentationTok}[1]{\textcolor[rgb]{0.73,0.13,0.13}{\textit{{#1}}}}
    \newcommand{\AnnotationTok}[1]{\textcolor[rgb]{0.38,0.63,0.69}{\textbf{\textit{{#1}}}}}
    \newcommand{\CommentVarTok}[1]{\textcolor[rgb]{0.38,0.63,0.69}{\textbf{\textit{{#1}}}}}
    \newcommand{\VariableTok}[1]{\textcolor[rgb]{0.10,0.09,0.49}{{#1}}}
    \newcommand{\ControlFlowTok}[1]{\textcolor[rgb]{0.00,0.44,0.13}{\textbf{{#1}}}}
    \newcommand{\OperatorTok}[1]{\textcolor[rgb]{0.40,0.40,0.40}{{#1}}}
    \newcommand{\BuiltInTok}[1]{{#1}}
    \newcommand{\ExtensionTok}[1]{{#1}}
    \newcommand{\PreprocessorTok}[1]{\textcolor[rgb]{0.74,0.48,0.00}{{#1}}}
    \newcommand{\AttributeTok}[1]{\textcolor[rgb]{0.49,0.56,0.16}{{#1}}}
    \newcommand{\InformationTok}[1]{\textcolor[rgb]{0.38,0.63,0.69}{\textbf{\textit{{#1}}}}}
    \newcommand{\WarningTok}[1]{\textcolor[rgb]{0.38,0.63,0.69}{\textbf{\textit{{#1}}}}}
    
    
    % Define a nice break command that doesn't care if a line doesn't already
    % exist.
    \def\br{\hspace*{\fill} \\* }
    % Math Jax compatability definitions
    \def\gt{>}
    \def\lt{<}
    % Document parameters
    \title{NSGAII}
    
    
    

    % Pygments definitions
    
\makeatletter
\def\PY@reset{\let\PY@it=\relax \let\PY@bf=\relax%
    \let\PY@ul=\relax \let\PY@tc=\relax%
    \let\PY@bc=\relax \let\PY@ff=\relax}
\def\PY@tok#1{\csname PY@tok@#1\endcsname}
\def\PY@toks#1+{\ifx\relax#1\empty\else%
    \PY@tok{#1}\expandafter\PY@toks\fi}
\def\PY@do#1{\PY@bc{\PY@tc{\PY@ul{%
    \PY@it{\PY@bf{\PY@ff{#1}}}}}}}
\def\PY#1#2{\PY@reset\PY@toks#1+\relax+\PY@do{#2}}

\expandafter\def\csname PY@tok@w\endcsname{\def\PY@tc##1{\textcolor[rgb]{0.73,0.73,0.73}{##1}}}
\expandafter\def\csname PY@tok@c\endcsname{\let\PY@it=\textit\def\PY@tc##1{\textcolor[rgb]{0.25,0.50,0.50}{##1}}}
\expandafter\def\csname PY@tok@cp\endcsname{\def\PY@tc##1{\textcolor[rgb]{0.74,0.48,0.00}{##1}}}
\expandafter\def\csname PY@tok@k\endcsname{\let\PY@bf=\textbf\def\PY@tc##1{\textcolor[rgb]{0.00,0.50,0.00}{##1}}}
\expandafter\def\csname PY@tok@kp\endcsname{\def\PY@tc##1{\textcolor[rgb]{0.00,0.50,0.00}{##1}}}
\expandafter\def\csname PY@tok@kt\endcsname{\def\PY@tc##1{\textcolor[rgb]{0.69,0.00,0.25}{##1}}}
\expandafter\def\csname PY@tok@o\endcsname{\def\PY@tc##1{\textcolor[rgb]{0.40,0.40,0.40}{##1}}}
\expandafter\def\csname PY@tok@ow\endcsname{\let\PY@bf=\textbf\def\PY@tc##1{\textcolor[rgb]{0.67,0.13,1.00}{##1}}}
\expandafter\def\csname PY@tok@nb\endcsname{\def\PY@tc##1{\textcolor[rgb]{0.00,0.50,0.00}{##1}}}
\expandafter\def\csname PY@tok@nf\endcsname{\def\PY@tc##1{\textcolor[rgb]{0.00,0.00,1.00}{##1}}}
\expandafter\def\csname PY@tok@nc\endcsname{\let\PY@bf=\textbf\def\PY@tc##1{\textcolor[rgb]{0.00,0.00,1.00}{##1}}}
\expandafter\def\csname PY@tok@nn\endcsname{\let\PY@bf=\textbf\def\PY@tc##1{\textcolor[rgb]{0.00,0.00,1.00}{##1}}}
\expandafter\def\csname PY@tok@ne\endcsname{\let\PY@bf=\textbf\def\PY@tc##1{\textcolor[rgb]{0.82,0.25,0.23}{##1}}}
\expandafter\def\csname PY@tok@nv\endcsname{\def\PY@tc##1{\textcolor[rgb]{0.10,0.09,0.49}{##1}}}
\expandafter\def\csname PY@tok@no\endcsname{\def\PY@tc##1{\textcolor[rgb]{0.53,0.00,0.00}{##1}}}
\expandafter\def\csname PY@tok@nl\endcsname{\def\PY@tc##1{\textcolor[rgb]{0.63,0.63,0.00}{##1}}}
\expandafter\def\csname PY@tok@ni\endcsname{\let\PY@bf=\textbf\def\PY@tc##1{\textcolor[rgb]{0.60,0.60,0.60}{##1}}}
\expandafter\def\csname PY@tok@na\endcsname{\def\PY@tc##1{\textcolor[rgb]{0.49,0.56,0.16}{##1}}}
\expandafter\def\csname PY@tok@nt\endcsname{\let\PY@bf=\textbf\def\PY@tc##1{\textcolor[rgb]{0.00,0.50,0.00}{##1}}}
\expandafter\def\csname PY@tok@nd\endcsname{\def\PY@tc##1{\textcolor[rgb]{0.67,0.13,1.00}{##1}}}
\expandafter\def\csname PY@tok@s\endcsname{\def\PY@tc##1{\textcolor[rgb]{0.73,0.13,0.13}{##1}}}
\expandafter\def\csname PY@tok@sd\endcsname{\let\PY@it=\textit\def\PY@tc##1{\textcolor[rgb]{0.73,0.13,0.13}{##1}}}
\expandafter\def\csname PY@tok@si\endcsname{\let\PY@bf=\textbf\def\PY@tc##1{\textcolor[rgb]{0.73,0.40,0.53}{##1}}}
\expandafter\def\csname PY@tok@se\endcsname{\let\PY@bf=\textbf\def\PY@tc##1{\textcolor[rgb]{0.73,0.40,0.13}{##1}}}
\expandafter\def\csname PY@tok@sr\endcsname{\def\PY@tc##1{\textcolor[rgb]{0.73,0.40,0.53}{##1}}}
\expandafter\def\csname PY@tok@ss\endcsname{\def\PY@tc##1{\textcolor[rgb]{0.10,0.09,0.49}{##1}}}
\expandafter\def\csname PY@tok@sx\endcsname{\def\PY@tc##1{\textcolor[rgb]{0.00,0.50,0.00}{##1}}}
\expandafter\def\csname PY@tok@m\endcsname{\def\PY@tc##1{\textcolor[rgb]{0.40,0.40,0.40}{##1}}}
\expandafter\def\csname PY@tok@gh\endcsname{\let\PY@bf=\textbf\def\PY@tc##1{\textcolor[rgb]{0.00,0.00,0.50}{##1}}}
\expandafter\def\csname PY@tok@gu\endcsname{\let\PY@bf=\textbf\def\PY@tc##1{\textcolor[rgb]{0.50,0.00,0.50}{##1}}}
\expandafter\def\csname PY@tok@gd\endcsname{\def\PY@tc##1{\textcolor[rgb]{0.63,0.00,0.00}{##1}}}
\expandafter\def\csname PY@tok@gi\endcsname{\def\PY@tc##1{\textcolor[rgb]{0.00,0.63,0.00}{##1}}}
\expandafter\def\csname PY@tok@gr\endcsname{\def\PY@tc##1{\textcolor[rgb]{1.00,0.00,0.00}{##1}}}
\expandafter\def\csname PY@tok@ge\endcsname{\let\PY@it=\textit}
\expandafter\def\csname PY@tok@gs\endcsname{\let\PY@bf=\textbf}
\expandafter\def\csname PY@tok@gp\endcsname{\let\PY@bf=\textbf\def\PY@tc##1{\textcolor[rgb]{0.00,0.00,0.50}{##1}}}
\expandafter\def\csname PY@tok@go\endcsname{\def\PY@tc##1{\textcolor[rgb]{0.53,0.53,0.53}{##1}}}
\expandafter\def\csname PY@tok@gt\endcsname{\def\PY@tc##1{\textcolor[rgb]{0.00,0.27,0.87}{##1}}}
\expandafter\def\csname PY@tok@err\endcsname{\def\PY@bc##1{\setlength{\fboxsep}{0pt}\fcolorbox[rgb]{1.00,0.00,0.00}{1,1,1}{\strut ##1}}}
\expandafter\def\csname PY@tok@kc\endcsname{\let\PY@bf=\textbf\def\PY@tc##1{\textcolor[rgb]{0.00,0.50,0.00}{##1}}}
\expandafter\def\csname PY@tok@kd\endcsname{\let\PY@bf=\textbf\def\PY@tc##1{\textcolor[rgb]{0.00,0.50,0.00}{##1}}}
\expandafter\def\csname PY@tok@kn\endcsname{\let\PY@bf=\textbf\def\PY@tc##1{\textcolor[rgb]{0.00,0.50,0.00}{##1}}}
\expandafter\def\csname PY@tok@kr\endcsname{\let\PY@bf=\textbf\def\PY@tc##1{\textcolor[rgb]{0.00,0.50,0.00}{##1}}}
\expandafter\def\csname PY@tok@bp\endcsname{\def\PY@tc##1{\textcolor[rgb]{0.00,0.50,0.00}{##1}}}
\expandafter\def\csname PY@tok@fm\endcsname{\def\PY@tc##1{\textcolor[rgb]{0.00,0.00,1.00}{##1}}}
\expandafter\def\csname PY@tok@vc\endcsname{\def\PY@tc##1{\textcolor[rgb]{0.10,0.09,0.49}{##1}}}
\expandafter\def\csname PY@tok@vg\endcsname{\def\PY@tc##1{\textcolor[rgb]{0.10,0.09,0.49}{##1}}}
\expandafter\def\csname PY@tok@vi\endcsname{\def\PY@tc##1{\textcolor[rgb]{0.10,0.09,0.49}{##1}}}
\expandafter\def\csname PY@tok@vm\endcsname{\def\PY@tc##1{\textcolor[rgb]{0.10,0.09,0.49}{##1}}}
\expandafter\def\csname PY@tok@sa\endcsname{\def\PY@tc##1{\textcolor[rgb]{0.73,0.13,0.13}{##1}}}
\expandafter\def\csname PY@tok@sb\endcsname{\def\PY@tc##1{\textcolor[rgb]{0.73,0.13,0.13}{##1}}}
\expandafter\def\csname PY@tok@sc\endcsname{\def\PY@tc##1{\textcolor[rgb]{0.73,0.13,0.13}{##1}}}
\expandafter\def\csname PY@tok@dl\endcsname{\def\PY@tc##1{\textcolor[rgb]{0.73,0.13,0.13}{##1}}}
\expandafter\def\csname PY@tok@s2\endcsname{\def\PY@tc##1{\textcolor[rgb]{0.73,0.13,0.13}{##1}}}
\expandafter\def\csname PY@tok@sh\endcsname{\def\PY@tc##1{\textcolor[rgb]{0.73,0.13,0.13}{##1}}}
\expandafter\def\csname PY@tok@s1\endcsname{\def\PY@tc##1{\textcolor[rgb]{0.73,0.13,0.13}{##1}}}
\expandafter\def\csname PY@tok@mb\endcsname{\def\PY@tc##1{\textcolor[rgb]{0.40,0.40,0.40}{##1}}}
\expandafter\def\csname PY@tok@mf\endcsname{\def\PY@tc##1{\textcolor[rgb]{0.40,0.40,0.40}{##1}}}
\expandafter\def\csname PY@tok@mh\endcsname{\def\PY@tc##1{\textcolor[rgb]{0.40,0.40,0.40}{##1}}}
\expandafter\def\csname PY@tok@mi\endcsname{\def\PY@tc##1{\textcolor[rgb]{0.40,0.40,0.40}{##1}}}
\expandafter\def\csname PY@tok@il\endcsname{\def\PY@tc##1{\textcolor[rgb]{0.40,0.40,0.40}{##1}}}
\expandafter\def\csname PY@tok@mo\endcsname{\def\PY@tc##1{\textcolor[rgb]{0.40,0.40,0.40}{##1}}}
\expandafter\def\csname PY@tok@ch\endcsname{\let\PY@it=\textit\def\PY@tc##1{\textcolor[rgb]{0.25,0.50,0.50}{##1}}}
\expandafter\def\csname PY@tok@cm\endcsname{\let\PY@it=\textit\def\PY@tc##1{\textcolor[rgb]{0.25,0.50,0.50}{##1}}}
\expandafter\def\csname PY@tok@cpf\endcsname{\let\PY@it=\textit\def\PY@tc##1{\textcolor[rgb]{0.25,0.50,0.50}{##1}}}
\expandafter\def\csname PY@tok@c1\endcsname{\let\PY@it=\textit\def\PY@tc##1{\textcolor[rgb]{0.25,0.50,0.50}{##1}}}
\expandafter\def\csname PY@tok@cs\endcsname{\let\PY@it=\textit\def\PY@tc##1{\textcolor[rgb]{0.25,0.50,0.50}{##1}}}

\def\PYZbs{\char`\\}
\def\PYZus{\char`\_}
\def\PYZob{\char`\{}
\def\PYZcb{\char`\}}
\def\PYZca{\char`\^}
\def\PYZam{\char`\&}
\def\PYZlt{\char`\<}
\def\PYZgt{\char`\>}
\def\PYZsh{\char`\#}
\def\PYZpc{\char`\%}
\def\PYZdl{\char`\$}
\def\PYZhy{\char`\-}
\def\PYZsq{\char`\'}
\def\PYZdq{\char`\"}
\def\PYZti{\char`\~}
% for compatibility with earlier versions
\def\PYZat{@}
\def\PYZlb{[}
\def\PYZrb{]}
\makeatother


    % Exact colors from NB
    \definecolor{incolor}{rgb}{0.0, 0.0, 0.5}
    \definecolor{outcolor}{rgb}{0.545, 0.0, 0.0}



    
    % Prevent overflowing lines due to hard-to-break entities
    \sloppy 
    % Setup hyperref package
    \hypersetup{
      breaklinks=true,  % so long urls are correctly broken across lines
      colorlinks=true,
      urlcolor=urlcolor,
      linkcolor=linkcolor,
      citecolor=citecolor,
      }
    % Slightly bigger margins than the latex defaults
    
    \geometry{verbose,tmargin=1in,bmargin=1in,lmargin=1in,rmargin=1in}
    
    

    \begin{document}
    
    
    \maketitle
    
    

    
    \hypertarget{algoritmo-nsgaii-para-resolver-el-problema-de-de-optimizaciuxf3n-multiobjetivo-que-determina-portafolios-de-cultivos-en-la-producciuxf3n-agruxedcola-a-pequeuxf1a-escala}{%
\section{Algoritmo NSGAII para resolver el problema de de optimización
multiobjetivo que determina portafolios de cultivos en la producción
agrícola a pequeña
escala}\label{algoritmo-nsgaii-para-resolver-el-problema-de-de-optimizaciuxf3n-multiobjetivo-que-determina-portafolios-de-cultivos-en-la-producciuxf3n-agruxedcola-a-pequeuxf1a-escala}}

el Algoritmo Genético Elitista de Ordenamiento No-dominado de segunda
generación (\emph{Nondominated Sorting in Genetic Algorithms type II,
NSGA-II}) desarrollado por Deh y otros (2002). El algoritmo es de
carácter poblacional puesto que genera un conjunto se soluciones
aleatorias \(P\) en el instante \(t\) y, mediante operaciones (en este
caso, el cruce y mutación de los algoritmos genéticos convencionales),
genera una segunda población. La población de hijos \(Q\) tiene el mismo
tamaño de la población \(P\). La suma de ambos conjuntos conforman la
población \(R\), con tamaño \(2*N\). Una vez creado el conjunto \(R\) se
procede organizar las soluciones que lo conforman originales (padres) y
aquellas derivadas de los operadores genéticos (hijos). Para esto se
compara la dominancia y pertenencia a los distintos frentes de Pareto
(Srinivas \& Deb, 1994), en conjunto con la distancia de apilamiento, la
cual evalúa las potenciales mejores soluciones en un mismo frente (Deb
et al., 2002). Una vez organizados los individuos, se seleccionan los
\(N\) mejores, de tal forma que la población final es del tamaño
original de los padres, es decir, se genera un nuevo conjunto: \(P_t+1\)
. El proceso de generación de individuos se realiza durante \(g\)
generaciones.

El algoritmo utilziado es programado en MATLAB 2017b y está dividido en
8 funciones que se ejecutan de acuerdo a la siguiente imagen:

    \hypertarget{algoritmo-base-nsgaii-nsgaii.m}{%
\subsection{Algoritmo base (NSGAII)
(NSGAII.m)}\label{algoritmo-base-nsgaii-nsgaii.m}}

El NSGAII (Non-dominated Sorting Genetic Algorithm II) es un algoritmo
de computación evolutiva multiobjetivo desarrollado por K Deb en 2012.
El código a continuación se basa de manera estructural (e.g.~el orden de
algunas funciones, el tipo de torneo, la generación de los frentes de
Pareto y el cálculo de la distancia de apilamiento) en el propuesto por
Kanpur Genetic Algorithm Labaratory y kindly, (información sobre el
algoritmo original puede ser consultada en:
http://www.iitk.ac.in/kangal/) y estructurado por Aravind Seshadri,
Copyright (c) 2009. El algortimo desarrollado presenta diversas
diferencias como: la estructura de variables de decisión ya que sus
límites inferiores y superiores no son ingresados manualmente, en vez de
ello, la función denominada `funcion\_objetivo' es un archivo guia
(\emph{script}) editable para cambiar la configuración del modelo (en
conjunto con los respectivos archivos \emph{.data}) además, teniendo en
cuenta la estructura del problema, la generación de cromosomas y la
descodificación se genera de manera independiente, es decir, se crean
variables donde se almacena los resultados de las funciones
\emph{fitness}. Por otra parte la funcion `operador\_genetico' se
desarrolla con un tipo de cruce y mutación diferente al propuesto por
Kanpur Genetic y, para ello, se proponen otras funciones para generar
las nueva poblaciones, entre otras.

A continuación se presenta el acuerdo de propiedad intelectual original:

Original algorithm NSGA-II was developed by researchers in Kanpur
Genetic Algorithm Labarotary and kindly visit their website for more
information http://www.iitk.ac.in/kangal/

Copyright (c) 2009, Aravind Seshadri All rights reserved.

Redistribution and use in source and binary forms, with or without
modification, are permitted provided that the following conditions are
met: * Redistributions of source code must retain the above copyright
notice, this list of conditions and the following disclaimer. *
Redistributions in binary form must reproduce the above copyright
notice, this list of conditions and the following disclaimer in the
documentation and/or other materials provided with the distribution

THIS SOFTWARE IS PROVIDED BY THE COPYRIGHT HOLDERS AND CONTRIBUTORS ``AS
IS'' AND ANY EXPRESS OR IMPLIED WARRANTIES, INCLUDING, BUT NOT LIMITED
TO, THE IMPLIED WARRANTIES OF MERCHANTABILITY AND FITNESS FOR A
PARTICULAR PURPOSE ARE DISCLAIMED. IN NO EVENT SHALL THE COPYRIGHT OWNER
OR CONTRIBUTORS BE LIABLE FOR ANY DIRECT, INDIRECT, INCIDENTAL, SPECIAL,
EXEMPLARY, OR CONSEQUENTIAL DAMAGES (INCLUDING, BUT NOT LIMITED TO,
PROCUREMENT OF SUBSTITUTE GOODS OR SERVICES; LOSS OF USE, DATA, OR
PROFITS; OR BUSINESS INTERRUPTION) HOWEVER CAUSED AND ON ANY THEORY OF
LIABILITY, WHETHER IN CONTRACT, STRICT LIABILITY, OR TORT (INCLUDING
NEGLIGENCE OR OTHERWISE) ARISING IN ANY WAY OUT OF THE USE OF THIS
SOFTWARE, EVEN IF ADVISED OF THE POSSIBILITY OF SUCH DAMAGE.

    \begin{Verbatim}[commandchars=\\\{\}]
{\color{incolor}In [{\color{incolor}1}]:} \PY{k}{function}\PY{+w}{ }[Cromosomas, Matriz\PYZus{}objetivos, orden\PYZus{}poblacion] ...
        \PY{p}{=}\PY{+w}{ }\PY{n+nf}{NSGAII}\PY{p}{(} poblacion,generaciones \PY{p}{)}
        \PY{c+cm}{\PYZpc{}\PYZob{}}
        \PY{c+cm}{la función NSGAII( poblacion,generaciones ) cuenta con dos parámetros de}
        \PY{c+cm}{entrada. poblacion y generaciones, donde poblacion indica la cantidad de}
        \PY{c+cm}{individuos (cromosomas) permitidos pro cada generacion. La variable}
        \PY{c+cm}{generaciones indica el número de veces que evolucionará la poblacion. Las}
        \PY{c+cm}{vaiables de salida son: Cromosomas (conjunto de individos pertenecientes}
        \PY{c+cm}{a la última generación), Matriz\PYZus{}objetivos (valor de las funciones}
        \PY{c+cm}{fitness) y orden\PYZus{}poblacion (ubicación de cada solución en el frente de}
        \PY{c+cm}{pareto y su distancia de apilameinto).}
        \PY{c+cm}{\PYZpc{}\PYZcb{}}
        
        \PY{c}{\PYZpc{} Verificación de parámetros del modelo}
        
        \PY{c}{\PYZpc{}A continuación se verifica que los parámetros de entrada cumplan con}
        \PY{c}{\PYZpc{}requisitos como tipo de variable y su magnitud.}
        \PY{k}{if} \PY{n}{nargin} \PY{o}{\PYZlt{}} \PY{l+m+mi}{2}
        \PY{n}{error}\PY{p}{(}\PY{l+s}{\PYZsq{}}\PY{l+s}{NSGA\PYZhy{}II: Por favor, Ingrese el tamaño de la población y el número de generaciones como argumentos de entrada.\PYZsq{}}\PY{p}{)}\PY{p}{;}
        \PY{k}{end}
        \PY{c}{\PYZpc{} Tipo de argumentos (numéricos)}
        \PY{k}{if} \PY{n}{isnumeric}\PY{p}{(}\PY{n}{poblacion}\PY{p}{)} \PY{o}{==} \PY{l+m+mi}{0} \PY{o}{||} \PY{n}{isnumeric}\PY{p}{(}\PY{n}{generaciones}\PY{p}{)} \PY{o}{==} \PY{l+m+mi}{0}
        \PY{n}{error}\PY{p}{(}\PY{l+s}{\PYZsq{}}\PY{l+s}{Ambos argumentos de entrada Pobalación (pop) y Número de generaciones (gen) deben ser de tipo entero\PYZsq{}}\PY{p}{)}\PY{p}{;}
        \PY{k}{end}
        \PY{c}{\PYZpc{} El tamaño mínimo de la población debe ser de 20 individuos}
        \PY{k}{if} \PY{n}{poblacion} \PY{o}{\PYZlt{}} \PY{l+m+mi}{20}
        \PY{n}{error}\PY{p}{(}\PY{l+s}{\PYZsq{}}\PY{l+s}{El tamaño mínimo de la población debe ser de 20 individuos\PYZsq{}}\PY{p}{)}\PY{p}{;}
        \PY{k}{end}
        \PY{c}{\PYZpc{} La cantidad mínima de generaciones es de 5}
        \PY{k}{if} \PY{n}{generaciones} \PY{o}{\PYZlt{}} \PY{l+m+mi}{5}
        \PY{n}{error}\PY{p}{(}\PY{l+s}{\PYZsq{}}\PY{l+s}{La cantidad mínima de generaciones es de 5\PYZsq{}}\PY{p}{)}\PY{p}{;}
        \PY{k}{end}
        
        \PY{k}{if} \PY{n}{isinteger}\PY{p}{(}\PY{n}{poblacion}\PY{p}{)} \PY{o}{==} \PY{l+m+mi}{0} \PY{o}{||} \PY{n}{isinteger}\PY{p}{(}\PY{n}{generaciones}\PY{p}{)}\PY{o}{==}\PY{l+m+mi}{0}
        \PY{n}{fprintf}\PY{p}{(}\PY{l+s}{\PYZsq{}}\PY{l+s}{Los valores deben ser enteros y, por tanto, para evitar errores son redondeados al entero inferior\PYZsq{}}\PY{p}{)}
        \PY{c}{\PYZpc{} Verificar que las entradas son enteras, a partir de un redondeo}
        \PY{n}{poblacion} \PY{p}{=} \PY{n+nb}{round}\PY{p}{(}\PY{n}{poblacion}\PY{p}{)}\PY{p}{;}
        \PY{n}{generaciones} \PY{p}{=} \PY{n+nb}{round}\PY{p}{(}\PY{n}{generaciones}\PY{p}{)}\PY{p}{;}
        \PY{k}{end}
        
        \PY{c}{\PYZpc{} Carga de la función objetivo}
        \PY{p}{[} \PY{n}{cant\PYZus{}objetivos}\PY{p}{,} \PY{n}{cant\PYZus{}variables}\PY{p}{,} \PY{c}{...}
        \PY{n}{cant\PYZus{}periodos} \PY{p}{,}\PY{n}{cant\PYZus{}productos}\PY{p}{,} \PY{n}{cant\PYZus{}lotes}\PY{p}{,} \PY{n}{precio\PYZus{}venta}\PY{p}{,} \PY{c}{...}
        \PY{n}{rendimiento}\PY{p}{,} \PY{n}{areas}\PY{p}{,} \PY{n}{demanda}\PY{p}{,}\PY{n}{familia\PYZus{}botanica}\PY{p}{,}\PY{n}{familia\PYZus{}venta}\PY{p}{,}\PY{n}{Covkkp}\PY{p}{]} \PY{c}{...}
        \PY{p}{=} \PY{n}{funcion\PYZus{}objetivo}\PY{p}{(}\PY{p}{)}\PY{p}{;}
        
        \PY{c}{\PYZpc{} Creación de los cromosomas iniciales}
        \PY{p}{[} \PY{n}{Cromosomas}\PY{p}{,}\PY{n}{Matriz\PYZus{}objetivos} \PY{p}{]} \PY{p}{=} \PY{n}{inicializar\PYZus{}cromosomas}\PY{p}{(}\PY{n}{poblacion}\PY{p}{,} \PY{c}{...}
        \PY{n}{cant\PYZus{}objetivos}\PY{p}{,} \PY{n}{cant\PYZus{}periodos}\PY{p}{,} \PY{n}{cant\PYZus{}productos}\PY{p}{,} \PY{n}{cant\PYZus{}lotes}\PY{p}{,}\PY{n}{precio\PYZus{}venta}\PY{p}{,} \PY{c}{...}
        \PY{n}{rendimiento}\PY{p}{,} \PY{n}{areas}\PY{p}{,} \PY{n}{demanda}\PY{p}{,}\PY{n}{familia\PYZus{}botanica}\PY{p}{,}\PY{n}{familia\PYZus{}venta}\PY{p}{,}\PY{n}{Covkkp}\PY{p}{)}\PY{p}{;}
        \PY{c}{\PYZpc{} Determinar dominancia de la solución inicial}
        \PY{p}{[} \PY{n}{Cromosomas}\PY{p}{,} \PY{n}{Matriz\PYZus{}objetivos}\PY{p}{,} \PY{n}{orden\PYZus{}poblacion} \PY{p}{]} \PY{p}{=} \PY{c}{...}
        \PY{n}{ordenar\PYZus{}poblacion}\PY{p}{(} \PY{n}{Cromosomas}\PY{p}{,} \PY{n}{Matriz\PYZus{}objetivos}\PY{p}{,}\PY{n}{cant\PYZus{}objetivos} \PY{p}{)}\PY{p}{;}
        
        \PY{c}{\PYZpc{} Comienza la evolución de las generaciones}
        
        \PY{k}{for} \PY{n}{gen} \PY{p}{=} \PY{l+m+mi}{1}\PY{p}{:} \PY{n}{generaciones}
        \PY{c+cm}{\PYZpc{}\PYZob{}}
        \PY{c+cm}{Los padres son seleccionados para la reproducción para generar}
        \PY{c+cm}{descendencia a partir de un torneo tipo binario basado en en comparar la}
        \PY{c+cm}{función fitness (Matriz\PYZus{}objetivo, fila 1), de ahí se generan lo padres}
        \PY{c+cm}{para realizar los diversos crices.}
        \PY{c+cm}{\PYZpc{}\PYZcb{}}
        \PY{c}{\PYZpc{} se crea la cantidad de grupos a enfrentarse}
        \PY{n}{pool} \PY{p}{=} \PY{n+nb}{round}\PY{p}{(}\PY{n}{poblacion}\PY{o}{/}\PY{l+m+mi}{2}\PY{p}{)}\PY{p}{;}
        \PY{c}{\PYZpc{}     se define el tipo de torneo}
        \PY{n}{torneo} \PY{p}{=} \PY{l+m+mi}{2}\PY{p}{;}
        \PY{c}{\PYZpc{}    Selección de los padres por torneo}
        \PY{p}{[}\PY{n}{Cromosomas\PYZus{}padres} \PY{p}{,}\PY{n}{criterio\PYZus{}evaluacion\PYZus{}padres}\PY{p}{,}\PY{n}{objetivos\PYZus{}padres}\PY{p}{]}\PY{p}{=} \PY{c}{...}
        \PY{n}{seleccion\PYZus{}por\PYZus{}torneo}\PY{p}{(}\PY{n}{Cromosomas}\PY{p}{,} \PY{n}{Matriz\PYZus{}objetivos}\PY{p}{,} \PY{c}{...}
        \PY{n}{orden\PYZus{}poblacion}\PY{p}{,} \PY{n}{pool}\PY{p}{,} \PY{n}{torneo}\PY{p}{)}\PY{p}{;}
        \PY{c+cm}{\PYZpc{}\PYZob{}}
        \PY{c+cm}{Se ajusta la probabilidad de mutación, esto implica que del total de}
        \PY{c+cm}{modificaciones genéticas un \PYZsq{}probabilidad\PYZus{}mutacion\PYZsq{} no se formará}
        \PY{c+cm}{mediante el cruce de padres, en vez de ello, un segmento de su}
        \PY{c+cm}{información genética (para este caso la producción en un lote al azar) se}
        \PY{c+cm}{volverá a computar}
        \PY{c+cm}{\PYZpc{}\PYZcb{}}
        \PY{n}{probabilidad\PYZus{}mutacion}\PY{p}{=}\PY{l+m+mf}{0.10}\PY{p}{;}
        \PY{c}{\PYZpc{}    Creación de los hijos mediante los operadores genéticos}
        
        \PY{p}{[}\PY{n}{Cromosomas\PYZus{}hijos}\PY{p}{,} \PY{n}{objetivos\PYZus{}hijos}\PY{p}{]}\PY{p}{=} \PY{n}{operador\PYZus{}genetico}\PY{p}{(} \PY{c}{...}
        \PY{n}{cant\PYZus{}objetivos}\PY{p}{,} \PY{n}{cant\PYZus{}periodos}\PY{p}{,} \PY{n}{cant\PYZus{}lotes}\PY{p}{,}\PY{n}{Cromosomas\PYZus{}padres}\PY{p}{,} \PY{c}{...}
        \PY{n}{probabilidad\PYZus{}mutacion}\PY{p}{,}\PY{n}{poblacion}\PY{p}{,}\PY{n}{cant\PYZus{}productos}\PY{p}{,}\PY{n}{Covkkp}\PY{p}{,}\PY{c}{...}
        \PY{n}{precio\PYZus{}venta}\PY{p}{,}\PY{n}{rendimiento}\PY{p}{,}\PY{n}{areas}\PY{p}{,}\PY{n}{demanda}\PY{p}{,}\PY{n}{familia\PYZus{}botanica}\PY{p}{,}\PY{c}{...}
        \PY{n}{familia\PYZus{}venta}\PY{p}{)}\PY{p}{;}
        \PY{c}{\PYZpc{} Se crea una población intermedia, conformada por las soluciones padres las soluciones hijas}
        \PY{c}{\PYZpc{} Soluciones}
        \PY{n}{Cromosomas\PYZus{}intermedios}\PY{p}{=}\PY{n+nb}{cat}\PY{p}{(}\PY{l+m+mi}{2}\PY{p}{,}\PY{n}{Cromosomas\PYZus{}padres}\PY{p}{,}\PY{n}{Cromosomas\PYZus{}hijos}\PY{p}{)}\PY{p}{;}
        \PY{c}{\PYZpc{}     valor de las funciones de ajuste}
        \PY{n}{Matriz\PYZus{}objetivos\PYZus{}intermedio}\PY{p}{=}\PY{n+nb}{cat}\PY{p}{(}\PY{l+m+mi}{2}\PY{p}{,}\PY{n}{objetivos\PYZus{}padres}\PY{p}{,}\PY{n}{objetivos\PYZus{}hijos}\PY{p}{)}\PY{p}{;}
        \PY{c}{\PYZpc{} Determinar dominancia de la solución intermedia}
        \PY{p}{[} \PY{n}{Cromosomas\PYZus{}intermedios}\PY{p}{,} \PY{n}{Matriz\PYZus{}objetivos\PYZus{}intermedio}\PY{p}{,} \PY{c}{...}
        \PY{n}{orden\PYZus{}poblacion\PYZus{}intermedio} \PY{p}{]} \PY{p}{=} \PY{n}{ordenar\PYZus{}poblacion}\PY{p}{(}\PY{c}{...}
        \PY{n}{Cromosomas\PYZus{}intermedios}\PY{p}{,} \PY{n}{Matriz\PYZus{}objetivos\PYZus{}intermedio}\PY{p}{,}\PY{n}{cant\PYZus{}objetivos} \PY{p}{)}\PY{p}{;}
        \PY{c+cm}{\PYZpc{}\PYZob{}}
        \PY{c+cm}{Se organizan las soluciones intermedias según el frente (de menor a}
        \PY{c+cm}{mayor) y según su distancia de apilamiento (de mayor a menor para el caso}
        \PY{c+cm}{de empatar en frente)}
        \PY{c+cm}{\PYZpc{}\PYZcb{}}
        \PY{c}{\PYZpc{}Se crea una variable auxiliar para almacenar el orden}
        \PY{n}{x}\PY{p}{=}\PY{n}{orden\PYZus{}poblacion\PYZus{}intermedio}\PY{o}{\PYZsq{}}\PY{p}{;}
        \PY{c}{\PYZpc{} Se crea una columna auxiliar con la posición de cada solución}
        \PY{n}{x}\PY{p}{(}\PY{p}{:}\PY{p}{,}\PY{l+m+mi}{3}\PY{p}{)}\PY{p}{=}\PY{l+m+mi}{1}\PY{p}{:}\PY{n+nb}{length}\PY{p}{(}\PY{n}{x}\PY{p}{)}\PY{p}{;}
        \PY{c}{\PYZpc{} Se ordena la solución según frente (columna 1) y según distancia (columna}
        \PY{c}{\PYZpc{} 2)}
        \PY{n}{orden}\PY{p}{=}\PY{n}{sortrows}\PY{p}{(}\PY{n}{x}\PY{p}{,}\PY{p}{[}\PY{l+m+mi}{1}\PY{p}{,}\PY{l+m+mi}{2}\PY{p}{]}\PY{p}{,}\PY{p}{\PYZob{}}\PY{l+s}{\PYZsq{}}\PY{l+s}{ascend\PYZsq{}}\PY{p}{,}\PY{l+s}{\PYZsq{}}\PY{l+s}{descend\PYZsq{}}\PY{p}{\PYZcb{}}\PY{p}{)}\PY{p}{;}
        \PY{c}{\PYZpc{} Se extrae el orden de las funciones y se almacena en la variable}
        \PY{c}{\PYZpc{} \PYZsq{}orden\PYZsq{}}
        \PY{n}{orden}\PY{p}{=}\PY{n}{orden}\PY{p}{(}\PY{p}{:}\PY{p}{,}\PY{l+m+mi}{3}\PY{p}{)}\PY{o}{\PYZsq{}}\PY{p}{;}
        \PY{c+cm}{\PYZpc{}\PYZob{}}
        \PY{c+cm}{Se actualiza la población \PYZsq{}Cromosomas\PYZsq{} según la población intermedia}
        \PY{c+cm}{\PYZsq{}Cromosomas\PYZus{}intermedios\PYZsq{} para las posiciones \PYZsq{}orden\PYZsq{} hasta un tamaño}
        \PY{c+cm}{máximo igual a la cantidad de individuos de la población \PYZsq{}poblacion\PYZsq{}}
        \PY{c+cm}{\PYZpc{}\PYZcb{}}
        
        \PY{k}{for} \PY{n}{ajuste}\PY{p}{=}\PY{l+m+mi}{1}\PY{p}{:}\PY{n+nb}{length}\PY{p}{(}\PY{n}{orden}\PY{p}{(}\PY{l+m+mi}{1}\PY{p}{:}\PY{n}{poblacion}\PY{p}{)}\PY{p}{)}
        \PY{n}{Cromosomas}\PY{p}{(}\PY{p}{:}\PY{p}{,}\PY{p}{(}\PY{n}{ajuste}\PY{o}{\PYZhy{}}\PY{l+m+mi}{1}\PY{p}{)}\PY{o}{*}\PY{n}{cant\PYZus{}lotes}\PY{o}{+}\PY{l+m+mi}{1}\PY{p}{:}\PY{p}{(}\PY{n}{ajuste}\PY{o}{\PYZhy{}}\PY{l+m+mi}{1}\PY{p}{)}\PY{o}{*}\PY{n}{cant\PYZus{}lotes}\PY{o}{+}\PY{n}{cant\PYZus{}lotes}\PY{p}{)}\PY{p}{=}\PY{c}{...}
        \PY{n}{Cromosomas\PYZus{}intermedios}\PY{p}{(}\PY{p}{:}\PY{p}{,}\PY{p}{(}\PY{n}{orden}\PY{p}{(}\PY{n}{ajuste}\PY{p}{)}\PY{o}{\PYZhy{}}\PY{l+m+mi}{1}\PY{p}{)}\PY{o}{*}\PY{n}{cant\PYZus{}lotes}\PY{o}{+}\PY{l+m+mi}{1}\PY{p}{:}\PY{c}{...}
        \PY{p}{(}\PY{n}{orden}\PY{p}{(}\PY{n}{ajuste}\PY{p}{)}\PY{o}{\PYZhy{}}\PY{l+m+mi}{1}\PY{p}{)}\PY{o}{*}\PY{n}{cant\PYZus{}lotes}\PY{o}{+}\PY{n}{cant\PYZus{}lotes}\PY{p}{)}\PY{p}{;}
        \PY{k}{end}
        \PY{c}{\PYZpc{} Se actualiza el valor de las funciones objetivo de la nueva población}
        \PY{n}{Matriz\PYZus{}objetivos}\PY{p}{=}\PY{n}{Matriz\PYZus{}objetivos\PYZus{}intermedio}\PY{p}{(}\PY{p}{:}\PY{p}{,}\PY{n}{orden}\PY{p}{(}\PY{l+m+mi}{1}\PY{p}{:}\PY{n}{poblacion}\PY{p}{)}\PY{p}{)}\PY{p}{;}
        \PY{n}{orden\PYZus{}poblacion}\PY{p}{=}\PY{n}{orden\PYZus{}poblacion\PYZus{}intermedio}\PY{p}{(}\PY{p}{:}\PY{p}{,}\PY{n}{orden}\PY{p}{(}\PY{l+m+mi}{1}\PY{p}{:}\PY{n}{poblacion}\PY{p}{)}\PY{p}{)}\PY{p}{;}
        \PY{k}{end}
        
        \PY{k}{end}
        \PY{k}{end}
\end{Verbatim}


    \begin{Verbatim}[commandchars=\\\{\}]
\textcolor{ansi-red-intense}{\textbf{Error: Function definitions are not permitted in this context.

}}
    \end{Verbatim}

    \hypertarget{algoritmo-para-definir-paruxe1metros-del-modelo-funcion_objetivo}{%
\subsection{Algoritmo para definir parámetros del modelo
(funcion\_objetivo)}\label{algoritmo-para-definir-paruxe1metros-del-modelo-funcion_objetivo}}

La función denominada ``funcion\_objetivo'', es un guión editable en el
cual se definen las dos funciones objetivo bajo estudio, así mismo, la
cantidad de variables y parámetros del modelo como: Número de periodos,
cantidad de Lotes, Cantidad de productos, número de variables de
decisión, cantidad de restricciones, etc. Teniendo en cuenta la
estructura del modelo, los parámetros son obtenidos a partir de de unos
datos almacenados en un archivo .data Para cambiar características como
el rendimiento agrícola por lote, el tamaño de cada lote, etc. Es
necesario cambiar el conjunto de datos.

Los datos de salida de la función son: FO\_cant\_objetivos (Cantidad de
objetivos), FO\_cant\_variables (Cantidad de
variables),FO\_cant\_periodos (Cantidad de periodos en el horizonte de
planeación),FO\_cant\_productos (cantidad de productos a cultivar y
recoger), FO\_cant\_lotes (cantidad de lotes a cultivar),
FO\_precio\_venta (precio de venta de cada producto en el horizonte de
planeación), FO\_rendimiento (Cantidad de kilogramos de cada producto a
recoger en cada lote), FO\_areas (tamaño en metros cuadrados de cada
lote), FO\_demanda (Demanda de cada familia de productos),
FO\_familia\_botanica (Familia botánica a la cual pertenece cada
producto),FO\_familia\_venta (Familia de productos sustitutos para
satisfacer la demanda)y FO\_Covkkp (covarianza de los rendimientos
económicos de cada producto).

    \begin{Verbatim}[commandchars=\\\{\}]
{\color{incolor}In [{\color{incolor} }]:} \PY{k}{function}\PY{err}{ }\PY{err}{[}\PY{n+nf}{FO\PYZus{}cant\PYZus{}objetivos}\PY{p}{,} \PY{n}{FO\PYZus{}cant\PYZus{}variables}\PY{p}{,}  \PY{c}{...}
        \PY{n}{FO\PYZus{}cant\PYZus{}periodos} \PY{p}{,}\PY{n}{FO\PYZus{}cant\PYZus{}productos}\PY{p}{,} \PY{n}{FO\PYZus{}cant\PYZus{}lotes}\PY{p}{,} \PY{n}{FO\PYZus{}precio\PYZus{}venta}\PY{p}{,} \PY{c}{...}
        \PY{n}{FO\PYZus{}rendimiento}\PY{p}{,} \PY{n}{FO\PYZus{}areas}\PY{p}{,} \PY{n}{FO\PYZus{}demanda}\PY{p}{,}\PY{n}{FO\PYZus{}familia\PYZus{}botanica}\PY{p}{,}\PY{c}{...}
        \PY{n}{FO\PYZus{}familia\PYZus{}venta}\PY{p}{,}\PY{n}{FO\PYZus{}Covkkp}\PY{p}{]} \PY{p}{=} \PY{n}{funcion\PYZus{}objetivo}\PY{p}{(}\PY{p}{)}
        \PY{c+cm}{\PYZpc{}\PYZob{}}
        \PY{c+cm}{Se cargan las características generales del problema a atender: 7039}
        \PY{c+cm}{Lotes, 21 Productos, 96 semanas para el horizonte de planeación, 5}
        \PY{c+cm}{familias botánicas, 4 categorías de producto de venta, La demanda para }
        \PY{c+cm}{cada categoría de venta, durante el horizonte de planeación.}
        \PY{c+cm}{\PYZpc{}\PYZcb{}}
        \PY{n}{load}\PY{p}{(}\PY{l+s}{\PYZsq{}}\PY{l+s}{productos\PYZus{}parametros\PYZsq{}}\PY{p}{)}
        \PY{c+cm}{\PYZpc{}\PYZob{}}
        \PY{c+cm}{Conjunto\PYZus{}s: Vector con los instantes donde puede sembrarse cada producto.}
        \PY{c+cm}{q:          Familia botánica a la que pertenece cada producto.}
        \PY{c+cm}{Ni:         Número de periodos que se demora el producto k en \PYZdq{}Madurar\PYZdq{}.}
        \PY{c+cm}{Pkt:        Precio de venta del producto K en el instante t}
        \PY{c+cm}{Rkl:        Rendimiento en Kg/m2 de cada producto (col) por lote (fil)}
        \PY{c+cm}{Al:         Área en metros de cada lote}
        \PY{c+cm}{Gv          Grupo de productos sustitutos para satisfacer una demanda}
        \PY{c+cm}{\PYZpc{}\PYZcb{}}
        \PY{c}{\PYZpc{} Se calcula la cantidad de periodos del cromosoma (en mes)}
        \PY{n}{T}\PY{p}{=}\PY{n+nb}{length}\PY{p}{(}\PY{n}{Conjunto\PYZus{}s}\PY{p}{)}\PY{o}{/}\PY{l+m+mi}{4}\PY{p}{;}
        \PY{c}{\PYZpc{} Cantidad de productos}
        \PY{n}{K}\PY{p}{=}\PY{n+nb}{length}\PY{p}{(}\PY{n}{Ni}\PY{p}{)}\PY{p}{;}
        \PY{c}{\PYZpc{} Cantidad de lotes para cada caso de prueba se trabajará con L lotes}
        \PY{n}{L}\PY{p}{=}\PY{l+m+mi}{40}\PY{p}{;}
        \PY{c}{\PYZpc{} Selecciono al azar L lotes}
        \PY{n}{lotes}\PY{p}{=}\PY{n}{datasample}\PY{p}{(}\PY{l+m+mi}{1}\PY{p}{:}\PY{n+nb}{length}\PY{p}{(}\PY{n}{Rkl}\PY{p}{(}\PY{p}{:}\PY{p}{,}\PY{l+m+mi}{1}\PY{p}{)}\PY{p}{)}\PY{p}{,}\PY{n}{L}\PY{p}{,}\PY{l+s}{\PYZsq{}}\PY{l+s}{Replace\PYZsq{}}\PY{p}{,}\PY{n}{false}\PY{p}{)}\PY{p}{;}
        \PY{c}{\PYZpc{} Actualizo los rendimientos para los dos lotes seleccionados}
        \PY{n}{Rkl}\PY{p}{=}\PY{n}{Rkl}\PY{p}{(}\PY{n}{lotes}\PY{p}{,}\PY{p}{:}\PY{p}{)}\PY{p}{;}
        \PY{c}{\PYZpc{} Calculo el área de los dos lotes}
        \PY{n}{Al}\PY{p}{=}\PY{n}{Al}\PY{p}{(}\PY{n}{lotes}\PY{o}{\PYZsq{}}\PY{p}{)}\PY{p}{;}
        
        \PY{c}{\PYZpc{} Se estima la matriz de Covarianza de los precios;}
        \PY{n}{Covkkp}\PY{p}{=}\PY{n}{cov}\PY{p}{(}\PY{n}{diff}\PY{p}{(}\PY{n+nb}{log}\PY{p}{(}\PY{n}{Pkt}\PY{p}{)}\PY{o}{\PYZsq{}}\PY{p}{,}\PY{l+m+mi}{1}\PY{p}{)}\PY{p}{)}\PY{p}{;}
        \PY{c}{\PYZpc{} Determinar la cantidad de variables de decisión para cada una de las}
        \PY{c}{\PYZpc{} familias de variables:}
        \PY{n}{Cant\PYZus{}Y}\PY{p}{=}\PY{n}{K}\PY{o}{*}\PY{n}{T}\PY{o}{*}\PY{n}{L}\PY{p}{;}
        \PY{n}{Cant\PYZus{}V}\PY{p}{=}\PY{n}{K}\PY{o}{*}\PY{n}{T}\PY{o}{*}\PY{n}{L}\PY{p}{;}
        \PY{n}{Cant\PYZus{}Z}\PY{p}{=}\PY{n}{K}\PY{o}{*}\PY{n}{T}\PY{o}{*}\PY{n}{L}\PY{p}{;}
        \PY{n}{Cant\PYZus{}U}\PY{p}{=}\PY{n}{K}\PY{o}{*}\PY{n}{K}\PY{o}{*}\PY{n}{T}\PY{o}{*}\PY{n}{L}\PY{p}{;}
        \PY{c}{\PYZpc{} Cantidad de variables}
        \PY{n}{FO\PYZus{}cant\PYZus{}variables}\PY{p}{=}\PY{n}{Cant\PYZus{}Y}\PY{o}{+}\PY{n}{Cant\PYZus{}V}\PY{o}{+}\PY{n}{Cant\PYZus{}Z}\PY{o}{+}\PY{n}{Cant\PYZus{}U}\PY{p}{;}
        \PY{c}{\PYZpc{} Cantidad de objetivos: por defecto son dos}
        \PY{n}{FO\PYZus{}cant\PYZus{}objetivos}\PY{p}{=}\PY{l+m+mi}{2}\PY{p}{;}
        \PY{c+cm}{\PYZpc{}\PYZob{}}
        \PY{c+cm}{Para este caso, la cantidad de periodos es de 24 meses, los cuales}
        \PY{c+cm}{corresponden a la aproximación de las 96 semanas de duración del}
        \PY{c+cm}{horizonte de planeación propuesto para el modelo exacto.}
        \PY{c+cm}{\PYZpc{}\PYZcb{}}
        \PY{n}{FO\PYZus{}cant\PYZus{}periodos}\PY{p}{=}\PY{n}{T}\PY{p}{;}
        \PY{n}{FO\PYZus{}cant\PYZus{}productos}\PY{p}{=}\PY{n}{K}\PY{p}{;}
        \PY{n}{FO\PYZus{}cant\PYZus{}lotes}\PY{p}{=}\PY{n}{L}\PY{p}{;}
        \PY{n}{FO\PYZus{}precio\PYZus{}venta}\PY{p}{=}\PY{n}{Pkt}\PY{p}{;}
        \PY{n}{FO\PYZus{}rendimiento}\PY{p}{=}\PY{n}{Rkl}\PY{p}{;}
        \PY{n}{FO\PYZus{}areas}\PY{p}{=}\PY{n}{Al}\PY{p}{;}
        \PY{n}{FO\PYZus{}demanda}\PY{p}{=}\PY{n}{Demanda\PYZus{}Gv}\PY{p}{;}
        \PY{n}{FO\PYZus{}familia\PYZus{}botanica}\PY{p}{=}\PY{n}{q}\PY{p}{;}
        \PY{n}{FO\PYZus{}familia\PYZus{}venta}\PY{p}{=}\PY{n}{Gv}\PY{p}{;}
        \PY{n}{FO\PYZus{}Covkkp}\PY{p}{=}\PY{n}{Covkkp}\PY{p}{;}
        \PY{k}{end}
\end{Verbatim}


    \hypertarget{algortimo-para-generar-los-cromosomas-iniciales-inicializar_cromosomas.m}{%
\subsection{Algortimo para generar los cromosomas iniciales
(inicializar\_cromosomas.m)}\label{algortimo-para-generar-los-cromosomas-iniciales-inicializar_cromosomas.m}}

La función denominada ``inicializar\_cromosomas'', tiene como parámetros
de entrada las características del modelo aproximado (con unidades de
tiempo basadas en meses en vez de semanas como el modelo exacto) y la
cantidad de individuos que conforman la población. Esta función carga
datos relacionados con las características del cromosoma como cantidad
de periodos y restricciones de siembra y demás, retorna los cromosomas y
sus respectivas funciones objetivo. Para cambiar el modelo, es necesario
modificar el archivo .data ya que este indica los periodos de siembra,
duración del periodo de madurez y otros elementos que caracterizan el
caso de estudio a trabajar.

Las variables de entrada son: poblacion (tamaño de la población),
cant\_objetivos (cantidad de objetivos a evaluar), cant\_periodos
(tamaño del horizonte del planeación), cant\_productos (cantidad de
productos a cultivar), cant\_lotes (cantidad de ltoes a ser cultivados),
precio\_venta (Precio de venta de cada producto para cada semana),
rendimiento (cantidad de kilogramos por metro cuadrado a recoger de cada
producto), areas (tamaño en metros cuadrados de cada lote), demanda
(Demanda de cada categoría de productos),familia\_botanica (Familia
botánica a la que pertenece cada producto),familia\_venta (Categorías o
familia de venta a la cual pertenece cada producto) y Covkkp (covarianza
entre los rendimientos o retornos económicos de cada producto).

Las variables de salida son: IC\_Cromosomas (Conjunto de cromosomas o
soluciones generadas de manera aleatoria) y IC\_Matriz\_objetivos
(Matriz de objetivos, la cual incluye en primera linea o fila la
penalización por inclumplimiento de restricciones, la segunda fila es el
primer objetivo (Maximizar ingresos) menos la penalización y la tercera
fila corresponde al segundo objetivo (Minimizar riesgo financiero)
penalizada (sumando) por el valor de la fila 1.

    \begin{Verbatim}[commandchars=\\\{\}]
{\color{incolor}In [{\color{incolor} }]:} \PY{k}{function}\PY{err}{ }\PY{err}{[}\PY{n+nf}{IC\PYZus{}Cromosomas}\PY{p}{,}\PY{n}{IC\PYZus{}Matriz\PYZus{}objetivos}\PY{p}{]} \PY{p}{=}\PY{c}{...}
        \PY{n}{inicializar\PYZus{}cromosomas}\PY{p}{(}\PY{n}{poblacion}\PY{p}{,} \PY{n}{cant\PYZus{}objetivos}\PY{p}{,} \PY{n}{cant\PYZus{}periodos}\PY{p}{,} \PY{c}{...}
        \PY{n}{cant\PYZus{}productos}\PY{p}{,} \PY{n}{cant\PYZus{}lotes}\PY{p}{,}\PY{n}{precio\PYZus{}venta}\PY{p}{,} \PY{n}{rendimiento}\PY{p}{,} \PY{n}{areas}\PY{p}{,} \PY{c}{...}
        \PY{n}{demanda}\PY{p}{,}\PY{n}{familia\PYZus{}botanica}\PY{p}{,}\PY{n}{familia\PYZus{}venta}\PY{p}{,}\PY{n}{Covkkp}\PY{p}{)}
        \PY{c+cm}{\PYZpc{}\PYZob{}}
        \PY{c+cm}{Se cargan en el sistema los parámetros relacionados con las restricciones}
        \PY{c+cm}{de producción, como los tiempo en los cuales se puede sembrar cada}
        \PY{c+cm}{producto, la duración del mismo. Todo lo anterior, para la estructura de}
        \PY{c+cm}{cromosoma, la cual está en meses y no semanas como el problema original.}
        \PY{c+cm}{\PYZpc{}\PYZcb{}}
        \PY{n}{load}\PY{p}{(}\PY{l+s}{\PYZsq{}}\PY{l+s}{parametros\PYZus{}maduracion.mat\PYZsq{}}\PY{p}{)}
        \PY{c+cm}{\PYZpc{}\PYZob{}}
        \PY{c+cm}{PPP = Productos Por Periodo}
        \PY{c+cm}{MTS = Matriz Tiempos Siembra}
        \PY{c+cm}{MFS = Matriz Fecha Siembra}
        \PY{c+cm}{PM  = Periodo de maduración}
        \PY{c+cm}{PMS = Periodo de maduración en semanas}
        \PY{c+cm}{\PYZpc{}\PYZcb{}}
        \PY{c}{\PYZpc{}Cálculo de variables y parámetros}
        \PY{c}{\PYZpc{}Duración del proyecto}
        \PY{n}{T}\PY{p}{=}\PY{n}{cant\PYZus{}periodos}\PY{p}{;}
        \PY{c}{\PYZpc{} Cantidad de Lotes}
        \PY{n}{L}\PY{p}{=}\PY{n}{cant\PYZus{}lotes}\PY{p}{;}
        \PY{c}{\PYZpc{} Cantidad de cromosomas}
        \PY{n}{C}\PY{p}{=}\PY{n}{poblacion}\PY{p}{;}
        \PY{c}{\PYZpc{} Se contruye la matriz donde se almacenarán la función de ajuste, seguida}
        \PY{c}{\PYZpc{} de las funciones objetivo}
        \PY{n}{IC\PYZus{}Matriz\PYZus{}objetivos}\PY{p}{=}\PY{n+nb}{zeros}\PY{p}{(}\PY{n}{cant\PYZus{}objetivos}\PY{o}{+}\PY{l+m+mi}{1}\PY{p}{,}\PY{n}{poblacion}\PY{p}{)}\PY{p}{;}
        \PY{c}{\PYZpc{} Creación del cromosoma}
        \PY{n}{IC\PYZus{}Cromosomas}\PY{p}{=}\PY{n+nb}{zeros}\PY{p}{(}\PY{n}{T}\PY{p}{,}\PY{n}{L}\PY{o}{*}\PY{n}{C}\PY{p}{)}\PY{p}{;}
        \PY{c+cm}{\PYZpc{}\PYZob{}}
        \PY{c+cm}{Cromosomas auxiliares para llevar la asignación de tiempos y bloqueo de}
        \PY{c+cm}{Ayuda a organizar los productos en los diversos periodos}
        \PY{c+cm}{\PYZpc{}\PYZcb{}} 
        \PY{n}{Cromosoma\PYZus{}aux1}\PY{p}{=}\PY{l+m+mi}{0}\PY{o}{*}\PY{n}{IC\PYZus{}Cromosomas}\PY{p}{;}
        \PY{c}{\PYZpc{} Almacena el contador de meses en que dura ocupado cada terreno}
        \PY{n}{Cromosoma\PYZus{}aux2}\PY{p}{=}\PY{n}{Cromosoma\PYZus{}aux1}\PY{p}{;}
        \PY{c}{\PYZpc{} Ayuda a verificar condiciones del cromosoma}
        \PY{n}{Cromosoma\PYZus{}aux3}\PY{p}{=}\PY{n}{Cromosoma\PYZus{}aux1}\PY{o}{+}\PY{l+m+mi}{1}\PY{p}{;}
        \PY{c}{\PYZpc{} Se realiza un bucle durante la duración del proyecto, por ahorro en}
        \PY{c}{\PYZpc{} recursos computacionales, no se tiene en cuenta los tres últimos periodos}
        \PY{c}{\PYZpc{} ya que ningún producto puede ser sembrado en ese instante:}
        \PY{k}{for} \PY{n}{tiempo}\PY{p}{=}\PY{l+m+mi}{1}\PY{p}{:}\PY{l+m+mi}{1}\PY{p}{:}\PY{n}{T}\PY{o}{\PYZhy{}}\PY{l+m+mi}{3}
        \PY{c}{\PYZpc{}Para cada periodo se asignan aletoriamente los 25 productos en L}
        \PY{c}{\PYZpc{}lotes, siempre y cuando estos puedan ser sembrados en ese instante y}
        \PY{c}{\PYZpc{}existan lotes disponibles}
        \PY{n}{IC\PYZus{}Cromosomas}\PY{p}{(}\PY{n}{tiempo}\PY{p}{,}\PY{p}{:}\PY{p}{)}\PY{p}{=} \PY{p}{(}\PY{n}{datasample}\PY{p}{(}\PY{n+nb}{find}\PY{p}{(}\PY{n}{MFS}\PY{p}{(}\PY{p}{:}\PY{p}{,}\PY{n}{tiempo}\PY{p}{)}\PY{o}{\PYZti{}=}\PY{l+m+mi}{0}\PY{p}{)}\PY{p}{,}\PY{n}{L}\PY{o}{*}\PY{n}{C}\PY{p}{)}\PY{o}{\PYZsq{}}\PY{p}{)}\PY{o}{.*}\PY{c}{...}
        \PY{n}{Cromosoma\PYZus{}aux3}\PY{p}{(}\PY{n}{tiempo}\PY{p}{,}\PY{p}{:}\PY{p}{)}\PY{p}{;}
        \PY{c}{\PYZpc{}Se contruye el vector auxiliar 1}
        \PY{n}{Cromosoma\PYZus{}aux1}\PY{p}{(}\PY{n}{tiempo}\PY{p}{,}\PY{p}{:}\PY{p}{)}\PY{p}{=}\PY{n}{IC\PYZus{}Cromosomas}\PY{p}{(}\PY{n}{tiempo}\PY{p}{,}\PY{p}{:}\PY{p}{)}\PY{p}{;}
        \PY{c}{\PYZpc{}Se crea un listado de los productos que pueden ser sembrados en cada}
        \PY{c}{\PYZpc{}periodo}
        \PY{n}{var\PYZus{}aux1}\PY{p}{=}\PY{n+nb}{find}\PY{p}{(}\PY{n}{MFS}\PY{p}{(}\PY{p}{:}\PY{p}{,}\PY{n}{tiempo}\PY{p}{)}\PY{o}{\PYZti{}=}\PY{l+m+mi}{0}\PY{p}{)}\PY{p}{;}
        \PY{c}{\PYZpc{}Se registra en el vector auxiliar la cantidad de periodos de}
        \PY{c}{\PYZpc{}maduración para cada producto}
        \PY{k}{for} \PY{n}{listado}\PY{p}{=}\PY{l+m+mi}{1}\PY{p}{:}\PY{n+nb}{length}\PY{p}{(}\PY{n}{var\PYZus{}aux1}\PY{p}{)}
        \PY{n}{Cromosoma\PYZus{}aux2}\PY{p}{(}\PY{n}{tiempo}\PY{p}{,}\PY{p}{:}\PY{p}{)} \PY{p}{=} \PY{n}{Cromosoma\PYZus{}aux2}\PY{p}{(}\PY{n}{tiempo}\PY{p}{,}\PY{p}{:}\PY{p}{)} \PY{o}{+} \PY{p}{(}\PY{n}{IC\PYZus{}Cromosomas}\PY{p}{(}\PY{n}{tiempo}\PY{p}{,}\PY{p}{:}\PY{p}{)}\PY{o}{==}\PY{c}{...}
        \PY{n}{ar\PYZus{}aux1}\PY{p}{(}\PY{n}{listado}\PY{p}{)}\PY{p}{)}\PY{o}{*}\PY{n}{PM}\PY{p}{(}\PY{n}{var\PYZus{}aux1}\PY{p}{(}\PY{n}{listado}\PY{p}{)}\PY{p}{)}\PY{p}{;}
        \PY{k}{end}
        \PY{c+cm}{\PYZpc{}\PYZob{}}
        \PY{c+cm}{Una vez asignado el primer cultivo, se recorren los vectores}
        \PY{c+cm}{auxiliares con el fin de evitar doble asignación mientras cada lote}
        \PY{c+cm}{está ocupado (durante el periodo de madurez de cada lote), para ello,}
        \PY{c+cm}{un vector auxiliar es el contador de madurez y una vez sea cero,}
        \PY{c+cm}{puede sembrarse otro cultivo.}
        \PY{c+cm}{\PYZpc{}\PYZcb{}}         
        \PY{k}{while} \PY{n}{tiempo} \PY{o}{\PYZgt{}} \PY{l+m+mi}{1}
        \PY{n}{Cromosoma\PYZus{}aux2}\PY{p}{(}\PY{n}{tiempo}\PY{p}{,}\PY{p}{:}\PY{p}{)} \PY{p}{=} \PY{n}{Cromosoma\PYZus{}aux2}\PY{p}{(}\PY{n}{tiempo}\PY{o}{\PYZhy{}}\PY{l+m+mi}{1}\PY{p}{,}\PY{p}{:}\PY{p}{)} \PY{o}{\PYZhy{}} \PY{l+m+mi}{1}\PY{p}{;}
        \PY{c}{\PYZpc{}Se asigna aleatoriamente los cultivos para el subconjunto de}
        \PY{c}{\PYZpc{}productos respectivo a cada mes.}
        \PY{n}{IC\PYZus{}Cromosomas}\PY{p}{(}\PY{n}{tiempo}\PY{p}{,}\PY{p}{:}\PY{p}{)}\PY{p}{=}\PY{p}{(}\PY{n}{datasample}\PY{p}{(}\PY{n+nb}{find}\PY{p}{(}\PY{n}{MFS}\PY{p}{(}\PY{p}{:}\PY{p}{,}\PY{n}{tiempo}\PY{p}{)}\PY{o}{\PYZti{}=}\PY{l+m+mi}{0}\PY{p}{)}\PY{p}{,}\PY{n}{L}\PY{o}{*}\PY{n}{C}\PY{p}{)}\PY{o}{\PYZsq{}}\PY{p}{)}\PY{o}{.*}\PY{c}{...}
        \PY{p}{(}\PY{n}{Cromosoma\PYZus{}aux2}\PY{p}{(}\PY{n}{tiempo}\PY{p}{,}\PY{p}{:}\PY{p}{)} \PY{o}{\PYZlt{}}\PY{l+m+mi}{1}\PY{p}{)}\PY{p}{;}
        \PY{c}{\PYZpc{}Se actualiza el valor del cromosoma auxiliar}
        \PY{n}{Cromosoma\PYZus{}aux1}\PY{p}{(}\PY{n}{tiempo}\PY{p}{,}\PY{p}{:}\PY{p}{)}\PY{p}{=}\PY{n}{IC\PYZus{}Cromosomas}\PY{p}{(}\PY{n}{tiempo}\PY{p}{,}\PY{p}{:}\PY{p}{)}\PY{p}{;}
        \PY{c}{\PYZpc{}Determino un listado de los productos que pueden ser sembrados}
        \PY{c}{\PYZpc{}en cada periodo}
        \PY{n}{var\PYZus{}aux1}\PY{p}{=}\PY{n+nb}{find}\PY{p}{(}\PY{n}{MFS}\PY{p}{(}\PY{p}{:}\PY{p}{,}\PY{n}{tiempo}\PY{p}{)}\PY{o}{\PYZti{}=}\PY{l+m+mi}{0}\PY{p}{)}\PY{p}{;}
        \PY{c}{\PYZpc{}Se bloquea la asignación de productos mientras esté el periodo de}
        \PY{c}{\PYZpc{}madurez}
        \PY{k}{for} \PY{n}{listado}\PY{p}{=}\PY{l+m+mi}{1}\PY{p}{:}\PY{n+nb}{length}\PY{p}{(}\PY{n}{var\PYZus{}aux1}\PY{p}{)}
        \PY{n}{Cromosoma\PYZus{}aux2}\PY{p}{(}\PY{n}{tiempo}\PY{p}{,}\PY{p}{:}\PY{p}{)} \PY{p}{=} \PY{n}{Cromosoma\PYZus{}aux2}\PY{p}{(}\PY{n}{tiempo}\PY{p}{,}\PY{p}{:}\PY{p}{)} \PY{o}{+} \PY{p}{(}\PY{n}{IC\PYZus{}Cromosomas}\PY{p}{(}\PY{n}{tiempo}\PY{p}{,}\PY{p}{:}\PY{p}{)}\PY{o}{==}\PY{c}{...}
        \PY{n}{var\PYZus{}aux1}\PY{p}{(}\PY{n}{listado}\PY{p}{)}\PY{p}{)}\PY{o}{*}\PY{n}{PM}\PY{p}{(}\PY{n}{var\PYZus{}aux1}\PY{p}{(}\PY{n}{listado}\PY{p}{)}\PY{p}{)}\PY{p}{;}
        \PY{k}{end}
        \PY{k}{break}
        \PY{k}{end}
        \PY{k}{end}
        \PY{c}{\PYZpc{} Se evalúan las soluciones usando otra función}
        \PY{p}{[}\PY{n}{IC\PYZus{}Cromosomas}\PY{p}{,}\PY{n}{IC\PYZus{}Matriz\PYZus{}objetivos}\PY{p}{]}\PY{p}{=}\PY{n}{Evaluar\PYZus{}individuos}\PY{p}{(}\PY{n}{IC\PYZus{}Cromosomas}\PY{p}{,}\PY{c}{...}
        \PY{n}{IC\PYZus{}Matriz\PYZus{}objetivos} \PY{p}{,} \PY{n}{poblacion}\PY{p}{,} \PY{n}{cant\PYZus{}productos}\PY{p}{,} \PY{n}{cant\PYZus{}lotes}\PY{p}{,} \PY{n}{PMS}\PY{p}{,} \PY{c}{...}
        \PY{n}{Covkkp}\PY{p}{,} \PY{n}{precio\PYZus{}venta}\PY{p}{,}\PY{n}{rendimiento}\PY{p}{,} \PY{n}{areas}\PY{p}{,} \PY{n}{demanda}\PY{p}{,}\PY{n}{familia\PYZus{}botanica}\PY{p}{,}\PY{c}{...}
        \PY{n}{familia\PYZus{}venta}\PY{p}{)}\PY{p}{;}
        \PY{k}{end}
\end{Verbatim}


    \hypertarget{algortimo-para-evaluar-los-cromosomas-iniciales-evaluar_individuos.m}{%
\subsection{Algortimo para evaluar los cromosomas iniciales
(Evaluar\_individuos.m)}\label{algortimo-para-evaluar-los-cromosomas-iniciales-evaluar_individuos.m}}

La función denominada ``Evaluar\_individuos'' toma como parámetros de
entrada la población de soluciones y una variable en la cual se
almacenarán los resultados de tres funciones: 1) Función de ajuste, 2)
Maximizar ingresos y 3) minimizar riesgo financiero del portafolio.
Dentro de la función los datos son transformados parcialmente con el
propósito de evaluar el cumplimiento de ciertas restricciones y valorar
el desempeño de las soluciones.

Se construye una variable en la cual se almacenarán tres valores para
cada individuo. La primera fila corresponde a la función fitness, la
cual es la suma de la penalizaciones relacionadas con el incumplimiento
de restricciones y la cantidad de periodos inproductivos del lote: 1)
restricciones\_tiempo (la cual cuantifica la cantidad de variables en
todos los lotes que exceden el horizonte de planeación), 2)
restricciones\_holgura (cuantifica cuántos periodos por encima o por
debajo del horizonte de planeación se encuentra), 3) sobre producción,
la cual no se almacena como variable sino cuantifica la cantidad total
de kilogramos de productos (por familia de venta) sembrados y recogidos,
que no pueden ser vendidos ya que no existe demanda), 4) rotaciones
(cuenta la cantidad de veces que quedaron sembrados de manera seguida
dos productos que pertenecen a la misma familia. La segunda fila
corresponde al valor de la primera función objetivo (Maximizar ingresos)
menos las penalizaciones calculadas en la primera fila. La tercera fila
corresponde al valor de la segunda función objetivo (Minimizar riesgo
financiero) más la penalización calculada en la fila 1.

Las variables de entrada son: EI\_Cromosomas,EI\_Matriz\_objetivos,
EI\_poblacion (tamaño de la población), EI\_cant\_productos(cantidad de
productos a cultivar), EI\_cant\_lotes (cantidad de lotes a ser
cultivados), EI\_PMS (Periodo de maduración en semanas), EI\_Covkkp
(covarianza entre los rendimientos o retornos económicos de cada
producto), EI\_precio\_venta (Precio de venta de cada producto para cada
semana), EI\_rendimiento (cantidad de kilogramos por metro cuadrado a
recoger de cada producto), EI\_areas (tamaño en metros cuadrados de cada
lote), EI\_demanda (Demanda de cada categoría de productos),
EI\_familia\_botanica (Familia botánica a la que pertenece cada
producto), EI\_familia\_venta (Categorías o familia de venta a la cual
pertenece cada producto).

    \begin{Verbatim}[commandchars=\\\{\}]
{\color{incolor}In [{\color{incolor} }]:} \PY{k}{function}\PY{err}{ }\PY{err}{[}\PY{n+nf}{EI\PYZus{}Cromosomas}\PY{p}{,}\PY{n}{EI\PYZus{}Matriz\PYZus{}objetivos}\PY{p}{]} \PY{p}{=} \PY{c}{...}
        \PY{n}{Evaluar\PYZus{}individuos}\PY{p}{(}\PY{n}{EI\PYZus{}Cromosomas}\PY{p}{,}\PY{n}{EI\PYZus{}Matriz\PYZus{}objetivos} \PY{p}{,} \PY{n}{EI\PYZus{}poblacion}\PY{p}{,}  \PY{c}{...}
        \PY{n}{EI\PYZus{}cant\PYZus{}productos}\PY{p}{,} \PY{n}{EI\PYZus{}cant\PYZus{}lotes}\PY{p}{,} \PY{n}{EI\PYZus{}PMS}\PY{p}{,} \PY{n}{EI\PYZus{}Covkkp}\PY{p}{,} \PY{n}{EI\PYZus{}precio\PYZus{}venta}\PY{p}{,}\PY{c}{...}
        \PY{n}{EI\PYZus{}rendimiento}\PY{p}{,} \PY{n}{EI\PYZus{}areas}\PY{p}{,} \PY{n}{EI\PYZus{}demanda}\PY{p}{,}\PY{n}{EI\PYZus{}familia\PYZus{}botanica}\PY{p}{,}\PY{n}{EI\PYZus{}familia\PYZus{}venta}\PY{p}{)}
        
        \PY{c}{\PYZpc{} Se calcula la cantidad de periodos del proyecto.}
        \PY{p}{[}\PY{n}{T}\PY{p}{,}\PY{o}{\PYZti{}}\PY{p}{]}\PY{p}{=}\PY{n+nb}{size}\PY{p}{(}\PY{n}{EI\PYZus{}Cromosomas}\PY{p}{)}\PY{p}{;}
        \PY{c}{\PYZpc{} Se realiza un recorrido para evaluar cada individuo}
        \PY{k}{for} \PY{n}{individuo}\PY{p}{=}\PY{l+m+mi}{1}\PY{p}{:}\PY{n}{EI\PYZus{}poblacion}
        \PY{c}{\PYZpc{} Se crean una matriz para almacenar la cantidad de producto (en}
        \PY{c}{\PYZpc{} kilogramos) que se recoge de cada producto para cada individuo.}
        \PY{n}{Matriz\PYZus{}productos}\PY{p}{=}\PY{n+nb}{zeros}\PY{p}{(}\PY{l+m+mi}{1}\PY{p}{,}\PY{n}{EI\PYZus{}cant\PYZus{}productos}\PY{p}{)}\PY{p}{;}
        \PY{c}{\PYZpc{} Se construye un vector para almacenar el volumen de producción, este}
        \PY{c}{\PYZpc{} vector posteriormente se contrasta con la demanda para el horizonte}
        \PY{c}{\PYZpc{} de planeación}
        \PY{n}{Matriz\PYZus{}venta}\PY{p}{=}\PY{n+nb}{zeros}\PY{p}{(}\PY{n+nb}{length}\PY{p}{(}\PY{n}{EI\PYZus{}demanda}\PY{p}{)}\PY{p}{,}\PY{l+m+mi}{1}\PY{p}{)}\PY{p}{;}
        \PY{c}{\PYZpc{} Se contruye una variable para contabilizar la cantidad de veces que}
        \PY{c}{\PYZpc{} son sembrados dos productos de manera consecutiva que pertenecen a la}
        \PY{c}{\PYZpc{} misma familia botánica.}
        \PY{n}{rotaciones}\PY{p}{=}\PY{l+m+mi}{0}\PY{p}{;}
        \PY{c}{\PYZpc{} Se genera un recorrido para todos los lotes de cada uno de los}
        \PY{c}{\PYZpc{} individuos.}
        \PY{k}{for} \PY{n}{lote}\PY{p}{=}\PY{l+m+mi}{1}\PY{p}{:}\PY{n}{EI\PYZus{}cant\PYZus{}lotes}
        \PY{c}{\PYZpc{} Se construye una variable auxiliar}
        \PY{n}{recorrido}\PY{p}{=}\PY{n}{EI\PYZus{}cant\PYZus{}lotes}\PY{o}{*}\PY{p}{(}\PY{n}{individuo}\PY{o}{\PYZhy{}}\PY{l+m+mi}{1}\PY{p}{)}\PY{o}{+}\PY{n}{lote}\PY{p}{;}
        
        \PY{n}{restricciones\PYZus{}tiempo}\PY{p}{=}\PY{l+m+mi}{0}\PY{p}{;}
        \PY{c}{\PYZpc{} Se determina si la solución excede el horizonte de planeación}
        \PY{k}{if} \PY{n}{sum}\PY{p}{(}\PY{n}{cumsum}\PY{p}{(}\PY{n}{EI\PYZus{}PMS}\PY{p}{(}\PY{n}{EI\PYZus{}Cromosomas}\PY{p}{(}\PY{n}{EI\PYZus{}Cromosomas}\PY{p}{(}\PY{p}{:}\PY{p}{,}\PY{n}{recorrido}\PY{p}{)}\PY{o}{\PYZti{}=}\PY{l+m+mi}{0}\PY{p}{,}\PY{n}{recorrido}\PY{p}{)}\PY{p}{)}\PY{o}{+}\PY{l+m+mi}{1}\PY{p}{)}\PY{o}{\PYZgt{}}\PY{n}{T}\PY{o}{*}\PY{l+m+mi}{4}\PY{p}{)}\PY{o}{\PYZgt{}}\PY{p}{=}\PY{l+m+mi}{1}
        \PY{c}{\PYZpc{} Determinar cuántos productos en cada lote exceden el horizonte de planeación}
        \PY{n}{restricciones\PYZus{}tiempo}\PY{p}{=}\PY{n}{sum}\PY{p}{(}\PY{n}{cumsum}\PY{p}{(}\PY{n}{EI\PYZus{}PMS}\PY{p}{(}\PY{n}{EI\PYZus{}Cromosomas}\PY{p}{(}\PY{n}{EI\PYZus{}Cromosomas}\PY{p}{(}\PY{p}{:}\PY{p}{,}\PY{n}{recorrido}\PY{p}{)}\PY{o}{\PYZti{}=}\PY{l+m+mi}{0}\PY{p}{,}\PY{n}{recorrido}\PY{p}{)}\PY{p}{)}\PY{o}{+}\PY{l+m+mi}{1}\PY{p}{)}\PY{o}{\PYZgt{}}\PY{n}{T}\PY{o}{*}\PY{l+m+mi}{4}\PY{p}{)}\PY{p}{;}
        \PY{c+cm}{\PYZpc{}\PYZob{}}
        \PY{c+cm}{transformo el valor de recogida de la última solución por el valor máximo}
        \PY{c+cm}{del horizonte de planeación, con el propósito de determinar el precio de}
        \PY{c+cm}{venta y calcular las penalizaicones por holguras, para ello construyo una}
        \PY{c+cm}{varible auxiliar:}
        \PY{c+cm}{\PYZpc{}\PYZcb{}}
        
        \PY{n}{periodo}\PY{p}{=}\PY{n}{cumsum}\PY{p}{(}\PY{n}{EI\PYZus{}PMS}\PY{p}{(}\PY{n}{EI\PYZus{}Cromosomas}\PY{p}{(}\PY{n}{EI\PYZus{}Cromosomas}\PY{p}{(}\PY{p}{:}\PY{p}{,}\PY{n}{recorrido}\PY{p}{)}\PY{o}{\PYZti{}=}\PY{l+m+mi}{0}\PY{p}{,}\PY{n}{recorrido}\PY{p}{)}\PY{p}{)}\PY{o}{+}\PY{l+m+mi}{1}\PY{p}{)}\PY{p}{;}
        \PY{c}{\PYZpc{}  reemplazo el valor del tiempo de recogida del último producto}
        \PY{n}{periodo}\PY{p}{(}\PY{n+nb}{find}\PY{p}{(}\PY{n}{periodo}\PY{o}{\PYZgt{}}\PY{n}{T}\PY{o}{*}\PY{l+m+mi}{4}\PY{p}{)}\PY{p}{)}\PY{p}{=}\PY{n}{T}\PY{o}{*}\PY{l+m+mi}{4}\PY{p}{;}
        \PY{c}{\PYZpc{} Almaceno el valor de la primera función objetivo (mazimizar ingresos) en}
        \PY{c}{\PYZpc{} la variable de salida:}
        \PY{n}{EI\PYZus{}Matriz\PYZus{}objetivos}\PY{p}{(}\PY{l+m+mi}{2}\PY{p}{,}\PY{n}{individuo}\PY{p}{)}\PY{p}{=}\PY{n}{EI\PYZus{}Matriz\PYZus{}objetivos}\PY{p}{(}\PY{l+m+mi}{2}\PY{p}{,}\PY{n}{individuo}\PY{p}{)}\PY{o}{+}\PY{c}{...}
        \PY{n}{sum}\PY{p}{(}\PY{n}{EI\PYZus{}rendimiento}\PY{p}{(}\PY{n}{lote}\PY{p}{,}\PY{n}{EI\PYZus{}Cromosomas}\PY{p}{(}\PY{n}{EI\PYZus{}Cromosomas}\PY{p}{(}\PY{p}{:}\PY{p}{,}\PY{n}{recorrido}\PY{p}{)}\PY{o}{\PYZti{}=}\PY{l+m+mi}{0}\PY{p}{,}\PY{n}{recorrido}\PY{p}{)}\PY{p}{)}\PY{o}{\PYZsq{}}\PY{o}{*}\PY{c}{...}
        \PY{n}{EI\PYZus{}areas}\PY{p}{(}\PY{n}{lote}\PY{p}{)}\PY{o}{.*}\PY{n+nb}{diag}\PY{p}{(}\PY{n}{EI\PYZus{}precio\PYZus{}venta}\PY{p}{(}\PY{n}{EI\PYZus{}Cromosomas}\PY{p}{(}\PY{n}{EI\PYZus{}Cromosomas}\PY{p}{(}\PY{p}{:}\PY{p}{,}\PY{n}{recorrido}\PY{p}{)}\PY{o}{\PYZti{}=}\PY{l+m+mi}{0}\PY{p}{,}\PY{n}{recorrido}\PY{p}{)}\PY{p}{,}\PY{n}{periodo}\PY{p}{)}\PY{p}{)}\PY{p}{)}\PY{p}{;}
        \PY{c}{\PYZpc{} para cada lote, se almacena la cantidad de producto cosechado}
        \PY{n}{Matriz\PYZus{}productos}\PY{p}{(}\PY{n}{EI\PYZus{}Cromosomas}\PY{p}{(}\PY{n}{EI\PYZus{}Cromosomas}\PY{p}{(}\PY{p}{:}\PY{p}{,}\PY{n}{recorrido}\PY{p}{)}\PY{o}{\PYZti{}=}\PY{l+m+mi}{0}\PY{p}{,}\PY{n}{recorrido}\PY{p}{)}\PY{o}{\PYZsq{}}\PY{p}{)}\PY{p}{=}\PY{c}{...}
        \PY{n}{Matriz\PYZus{}productos}\PY{p}{(}\PY{n}{EI\PYZus{}Cromosomas}\PY{p}{(}\PY{n}{EI\PYZus{}Cromosomas}\PY{p}{(}\PY{p}{:}\PY{p}{,}\PY{n}{recorrido}\PY{p}{)}\PY{o}{\PYZti{}=}\PY{l+m+mi}{0}\PY{p}{,}\PY{n}{recorrido}\PY{p}{)}\PY{o}{\PYZsq{}}\PY{p}{)}\PY{o}{+}\PY{c}{...}
        \PY{p}{(}\PY{n}{EI\PYZus{}rendimiento}\PY{p}{(}\PY{n}{lote}\PY{p}{,}\PY{n}{EI\PYZus{}Cromosomas}\PY{p}{(}\PY{n}{EI\PYZus{}Cromosomas}\PY{p}{(}\PY{p}{:}\PY{p}{,}\PY{n}{recorrido}\PY{p}{)}\PY{o}{\PYZti{}=}\PY{l+m+mi}{0}\PY{p}{,}\PY{n}{recorrido}\PY{p}{)}\PY{p}{)}\PY{o}{\PYZsq{}}\PY{o}{*}\PY{n}{EI\PYZus{}areas}\PY{p}{(}\PY{n}{lote}\PY{p}{)}\PY{p}{)}\PY{o}{\PYZsq{}}\PY{p}{;}
        \PY{k}{else}
        \PY{n}{EI\PYZus{}Matriz\PYZus{}objetivos}\PY{p}{(}\PY{l+m+mi}{2}\PY{p}{,}\PY{n}{individuo}\PY{p}{)}\PY{p}{=}\PY{n}{EI\PYZus{}Matriz\PYZus{}objetivos}\PY{p}{(}\PY{l+m+mi}{2}\PY{p}{,}\PY{n}{individuo}\PY{p}{)}\PY{o}{+}\PY{c}{...}
        \PY{n}{sum}\PY{p}{(}\PY{n}{EI\PYZus{}rendimiento}\PY{p}{(}\PY{n}{lote}\PY{p}{,}\PY{n}{EI\PYZus{}Cromosomas}\PY{p}{(}\PY{n}{EI\PYZus{}Cromosomas}\PY{p}{(}\PY{p}{:}\PY{p}{,}\PY{n}{recorrido}\PY{p}{)}\PY{o}{\PYZti{}=}\PY{l+m+mi}{0}\PY{p}{,}\PY{n}{recorrido}\PY{p}{)}\PY{p}{)}\PY{o}{\PYZsq{}}\PY{o}{*}\PY{c}{...}
        \PY{n}{EI\PYZus{}areas}\PY{p}{(}\PY{n}{lote}\PY{p}{)}\PY{o}{.*}\PY{n+nb}{diag}\PY{p}{(}\PY{n}{EI\PYZus{}precio\PYZus{}venta}\PY{p}{(}\PY{n}{EI\PYZus{}Cromosomas}\PY{p}{(}\PY{n}{EI\PYZus{}Cromosomas}\PY{p}{(}\PY{p}{:}\PY{p}{,}\PY{n}{recorrido}\PY{p}{)}\PY{o}{\PYZti{}=}\PY{l+m+mi}{0}\PY{p}{,}\PY{n}{recorrido}\PY{p}{)}\PY{p}{,}\PY{c}{...}
        \PY{n}{cumsum}\PY{p}{(}\PY{n}{EI\PYZus{}PMS}\PY{p}{(}\PY{n}{EI\PYZus{}Cromosomas}\PY{p}{(}\PY{n}{EI\PYZus{}Cromosomas}\PY{p}{(}\PY{p}{:}\PY{p}{,}\PY{n}{recorrido}\PY{p}{)}\PY{o}{\PYZti{}=}\PY{l+m+mi}{0}\PY{p}{,}\PY{n}{recorrido}\PY{p}{)}\PY{p}{)}\PY{o}{+}\PY{l+m+mi}{1}\PY{p}{)}\PY{p}{)}\PY{p}{)}\PY{p}{)}\PY{p}{;}
        \PY{c}{\PYZpc{} para cada lote, se almacena la cantidad de producto cosechado}
        \PY{n}{Matriz\PYZus{}productos}\PY{p}{(}\PY{n}{EI\PYZus{}Cromosomas}\PY{p}{(}\PY{n}{EI\PYZus{}Cromosomas}\PY{p}{(}\PY{p}{:}\PY{p}{,}\PY{n}{recorrido}\PY{p}{)}\PY{o}{\PYZti{}=}\PY{l+m+mi}{0}\PY{p}{,}\PY{n}{recorrido}\PY{p}{)}\PY{o}{\PYZsq{}}\PY{p}{)}\PY{p}{=}\PY{c}{...}
        \PY{n}{Matriz\PYZus{}productos}\PY{p}{(}\PY{n}{EI\PYZus{}Cromosomas}\PY{p}{(}\PY{n}{EI\PYZus{}Cromosomas}\PY{p}{(}\PY{p}{:}\PY{p}{,}\PY{n}{recorrido}\PY{p}{)}\PY{o}{\PYZti{}=}\PY{l+m+mi}{0}\PY{p}{,}\PY{n}{recorrido}\PY{p}{)}\PY{o}{\PYZsq{}}\PY{p}{)}\PY{o}{+}\PY{c}{...}
        \PY{p}{(}\PY{n}{EI\PYZus{}rendimiento}\PY{p}{(}\PY{n}{lote}\PY{p}{,}\PY{n}{EI\PYZus{}Cromosomas}\PY{p}{(}\PY{n}{EI\PYZus{}Cromosomas}\PY{p}{(}\PY{p}{:}\PY{p}{,}\PY{n}{recorrido}\PY{p}{)}\PY{o}{\PYZti{}=}\PY{l+m+mi}{0}\PY{p}{,}\PY{n}{recorrido}\PY{p}{)}\PY{p}{)}\PY{o}{\PYZsq{}}\PY{o}{*}\PY{n}{EI\PYZus{}areas}\PY{p}{(}\PY{n}{lote}\PY{p}{)}\PY{p}{)}\PY{o}{\PYZsq{}}\PY{p}{;}
        \PY{k}{end}
        \PY{c}{\PYZpc{} Se cuantifican la cantidad de periodos de sub utilziación o sobre}
        \PY{c}{\PYZpc{} utilización del terreno:}
        \PY{n}{restricciones\PYZus{}holgura}\PY{p}{=}\PY{n+nb}{abs}\PY{p}{(}\PY{p}{(}\PY{n}{T}\PY{o}{*}\PY{l+m+mi}{4}\PY{o}{\PYZhy{}}\PY{n}{max}\PY{p}{(}\PY{n}{cumsum}\PY{p}{(}\PY{n}{EI\PYZus{}PMS}\PY{p}{(}\PY{n}{EI\PYZus{}Cromosomas}\PY{p}{(}\PY{n}{EI\PYZus{}Cromosomas}\PY{p}{(}\PY{p}{:}\PY{p}{,}\PY{n}{recorrido}\PY{p}{)}\PY{o}{\PYZti{}=}\PY{l+m+mi}{0}\PY{p}{,}\PY{c}{...}
        \PY{n}{recorrido}\PY{p}{)}\PY{p}{)}\PY{o}{+}\PY{l+m+mi}{1}\PY{p}{)}\PY{p}{)}\PY{p}{)}\PY{p}{)}\PY{p}{;}
        \PY{c}{\PYZpc{} Se almacena la cantidad de kilos recogida de cada producto}
        \PY{n}{kilos}\PY{p}{=} \PY{n}{EI\PYZus{}rendimiento}\PY{p}{(}\PY{n}{lote}\PY{p}{,}\PY{n}{EI\PYZus{}Cromosomas}\PY{p}{(}\PY{n}{EI\PYZus{}Cromosomas}\PY{p}{(}\PY{p}{:}\PY{p}{,}\PY{n}{recorrido}\PY{p}{)}\PY{o}{\PYZti{}=}\PY{l+m+mi}{0}\PY{p}{,}\PY{n}{recorrido}\PY{p}{)}\PY{p}{)}\PY{o}{\PYZsq{}}\PY{o}{*}\PY{n}{EI\PYZus{}areas}\PY{p}{(}\PY{n}{lote}\PY{p}{)}\PY{p}{;}
        \PY{c}{\PYZpc{} ¿A cuál grupo de venta pertenece cada producto?}
        \PY{n}{recorrido2}\PY{p}{=}\PY{n}{EI\PYZus{}familia\PYZus{}venta}\PY{p}{(}\PY{n}{EI\PYZus{}Cromosomas}\PY{p}{(}\PY{n}{EI\PYZus{}Cromosomas}\PY{p}{(}\PY{p}{:}\PY{p}{,}\PY{n}{recorrido}\PY{p}{)}\PY{o}{\PYZti{}=}\PY{l+m+mi}{0}\PY{p}{,}\PY{n}{recorrido}\PY{p}{)}\PY{p}{)}\PY{p}{;}
        \PY{c}{\PYZpc{} Se suma la cantidad de veces que no existe rotación de familias}
        \PY{c}{\PYZpc{} botánicas:}
        \PY{n}{rotaciones}\PY{p}{=}\PY{n}{rotaciones}\PY{o}{+}\PY{n}{sum}\PY{p}{(}\PY{n}{recorrido2}\PY{p}{(}\PY{l+m+mi}{2}\PY{p}{:}\PY{n+nb}{length}\PY{p}{(}\PY{n}{recorrido2}\PY{p}{)}\PY{p}{)}\PY{o}{==}\PY{n}{recorrido2}\PY{p}{(}\PY{l+m+mi}{1}\PY{p}{:}\PY{n+nb}{length}\PY{p}{(}\PY{n}{recorrido2}\PY{p}{)}\PY{o}{\PYZhy{}}\PY{l+m+mi}{1}\PY{p}{)}\PY{p}{)}\PY{p}{;}
        \PY{c}{\PYZpc{} Se crea un bucle para almacenar la producción en las respectivas}
        \PY{c}{\PYZpc{} familias de venta, con el fin de contrastar con la demanda}
        \PY{k}{for} \PY{n}{productos\PYZus{}vendidos}\PY{p}{=}\PY{l+m+mi}{1}\PY{p}{:}\PY{n+nb}{length}\PY{p}{(}\PY{n}{recorrido2}\PY{p}{)}
        \PY{c}{\PYZpc{} Se actualiza el valor de la matriz venta}
        \PY{n}{Matriz\PYZus{}venta}\PY{p}{(}\PY{n}{recorrido2}\PY{p}{(}\PY{n}{productos\PYZus{}vendidos}\PY{p}{)}\PY{p}{)}\PY{p}{=}\PY{n}{Matriz\PYZus{}venta}\PY{p}{(}\PY{n}{recorrido2}\PY{p}{(}\PY{n}{productos\PYZus{}vendidos}\PY{p}{)}\PY{p}{)}\PY{o}{+}\PY{c}{...}
        \PY{n}{kilos}\PY{p}{(}\PY{n}{productos\PYZus{}vendidos}\PY{p}{)}\PY{p}{;}
        \PY{k}{end}
        \PY{c}{\PYZpc{} Se evaluan variables una vez finalizado el recorrido por todos}
        \PY{c}{\PYZpc{} los lotes para cada ndividuo.}
        \PY{k}{if} \PY{n}{lote}\PY{o}{==}\PY{n}{EI\PYZus{}cant\PYZus{}lotes}
        \PY{c}{\PYZpc{} Se actualiza el valor de la función fitness}
        \PY{n}{EI\PYZus{}Matriz\PYZus{}objetivos}\PY{p}{(}\PY{l+m+mi}{1}\PY{p}{,}\PY{n}{individuo}\PY{p}{)}\PY{p}{=} \PY{o}{\PYZhy{}}\PY{c}{...}
        \PY{n}{restricciones\PYZus{}tiempo} \PY{o}{\PYZhy{}}\PY{c}{...}
        \PY{n}{restricciones\PYZus{}holgura} \PY{o}{\PYZhy{}}\PY{c}{...}
        \PY{n}{sum}\PY{p}{(}\PY{n}{Matriz\PYZus{}venta}\PY{o}{\PYZhy{}}\PY{n}{EI\PYZus{}demanda}\PY{o}{.*}\PY{n}{Matriz\PYZus{}venta}\PY{o}{\PYZgt{}}\PY{n}{EI\PYZus{}demanda}\PY{p}{)} \PY{o}{\PYZhy{}} \PY{c}{...}
        \PY{n}{rotaciones}\PY{p}{;}
        \PY{n}{EI\PYZus{}Matriz\PYZus{}objetivos}\PY{p}{(}\PY{l+m+mi}{2}\PY{p}{,}\PY{n}{individuo}\PY{p}{)}\PY{p}{=}\PY{n}{EI\PYZus{}Matriz\PYZus{}objetivos}\PY{p}{(}\PY{l+m+mi}{1}\PY{p}{,}\PY{n}{individuo}\PY{p}{)}\PY{o}{+}\PY{n}{EI\PYZus{}Matriz\PYZus{}objetivos}\PY{p}{(}\PY{l+m+mi}{2}\PY{p}{,}\PY{n}{individuo}\PY{p}{)}\PY{p}{;}
        \PY{c}{\PYZpc{} Se calcula el riesgo del portafolio}
        \PY{n}{EI\PYZus{}Matriz\PYZus{}objetivos}\PY{p}{(}\PY{l+m+mi}{3}\PY{p}{,}\PY{n}{individuo}\PY{p}{)}\PY{p}{=}\PY{n}{Matriz\PYZus{}productos}\PY{o}{*}\PY{n}{EI\PYZus{}Covkkp}\PY{o}{*}\PY{n}{Matriz\PYZus{}productos}\PY{o}{\PYZsq{}}\PY{o}{+}\PY{n}{EI\PYZus{}Matriz\PYZus{}objetivos}\PY{p}{(}\PY{l+m+mi}{1}\PY{p}{,}\PY{n}{individuo}\PY{p}{)}\PY{p}{;}
        \PY{k}{end}
        \PY{k}{end}
        \PY{k}{end}
        \PY{k}{end}
\end{Verbatim}


    \hypertarget{algortimo-para-determinar-la-dominancia-de-las-soluciones-ordenar_poblacion.m}{%
\subsection{Algortimo para determinar la dominancia de las soluciones
(ordenar\_poblacion.m)}\label{algortimo-para-determinar-la-dominancia-de-las-soluciones-ordenar_poblacion.m}}

La función clasifica la población actual basada en la no dominancia.
Todos los individuos en el primer frente reciben un rango de 1, a los
individuos del segundo frente se les asigna el rango 2 y así
sucesivamente. Después de asignar el rango se calcula la distancia de
apilamiento. las variables de entrada del modelo son: OP\_Cromosomas
(Soluciones a ordenar), OP\_Matriz\_objetivos (Funciones de ajuste) y
OP\_cant\_objetivos (Cantidad de objetivos). Como variables de salida
tiene dos soluciones sin transformar: OP\_Cromosomas y
OP\_Matriz\_objetivos. Además la variable OP\_criterio\_evaluacion
(criterios de evaluación, donde la primera fila corresponde al frente al
cual pertenece cada solución y la segunda fila la distancia de
apilamiento de la solución en su respectivo frente).

    \begin{Verbatim}[commandchars=\\\{\}]
{\color{incolor}In [{\color{incolor} }]:} \PY{k}{function}\PY{err}{ }\PY{err}{[}  \PY{n+nf}{OP\PYZus{}Cromosomas}\PY{p}{,} \PY{n}{OP\PYZus{}Matriz\PYZus{}objetivos}\PY{p}{,} \PY{n}{OP\PYZus{}criterio\PYZus{}evaluacion} \PY{p}{]} \PY{p}{=} \PY{c}{...}
        \PY{n}{ordenar\PYZus{}poblacion}\PY{p}{(}  \PY{n}{OP\PYZus{}Cromosomas}\PY{p}{,} \PY{n}{OP\PYZus{}Matriz\PYZus{}objetivos}\PY{p}{,} \PY{n}{OP\PYZus{}cant\PYZus{}objetivos} \PY{p}{)}
        \PY{c}{\PYZpc{} Se cargan las variables en la función}
        \PY{n}{OP\PYZus{}Cromosomas}\PY{p}{=}\PY{n}{OP\PYZus{}Cromosomas}\PY{p}{;}
        \PY{n}{OP\PYZus{}cant\PYZus{}objetivos}\PY{p}{=}\PY{n}{OP\PYZus{}cant\PYZus{}objetivos}\PY{p}{;}
        \PY{c}{\PYZpc{}Para comodidad de cálculos, se almacena el conjunto de valores de las}
        \PY{c}{\PYZpc{} funciones objetivo en una variable auxiliar.}
        \PY{n}{x}\PY{p}{=}\PY{n}{OP\PYZus{}Matriz\PYZus{}objetivos}\PY{o}{\PYZsq{}}\PY{p}{;}
        \PY{c}{\PYZpc{}Para el ordenamiento se utiliza solamente los objetivos de maximizar}
        \PY{c}{\PYZpc{}ingresos y disminuir riesgo.}
        \PY{n}{x}\PY{p}{=}\PY{n}{x}\PY{p}{(}\PY{p}{:}\PY{p}{,}\PY{l+m+mi}{2}\PY{p}{:}\PY{n}{OP\PYZus{}cant\PYZus{}objetivos}\PY{o}{+}\PY{l+m+mi}{1}\PY{p}{)}\PY{p}{;}
        \PY{c}{\PYZpc{}por comodidad de cálculo, los valores de la función de maximizar son}
        \PY{c}{\PYZpc{}premultiplicados por \PYZhy{}1 con el propósito de codificar el cálculo en}
        \PY{c}{\PYZpc{}funcion de variables de minimización}
        \PY{n}{x}\PY{p}{(}\PY{p}{:}\PY{p}{,}\PY{l+m+mi}{1}\PY{p}{)}\PY{p}{=}\PY{o}{\PYZhy{}}\PY{n}{x}\PY{p}{(}\PY{p}{:}\PY{p}{,}\PY{l+m+mi}{1}\PY{p}{)}\PY{p}{;}
        \PY{c}{\PYZpc{} se extrae la cantidad de individuos a organizar}
        \PY{p}{[}\PY{n}{N}\PY{p}{,} \PY{o}{\PYZti{}}\PY{p}{]} \PY{p}{=} \PY{n+nb}{size}\PY{p}{(}\PY{n}{x}\PY{p}{)}\PY{p}{;}
        \PY{c}{\PYZpc{} se utiliza una variable auxiliar para hacr cálculos con los frnetes de}
        \PY{c}{\PYZpc{} cada objetivo (se reemplaza ocn el fin de facilitar la escritura dle}
        \PY{c}{\PYZpc{} código)}
        \PY{n}{M}\PY{p}{=}\PY{n}{OP\PYZus{}cant\PYZus{}objetivos}\PY{p}{;}
        \PY{c}{\PYZpc{} Se inicializa un contador de frentes de pareto}
        \PY{n}{frente} \PY{p}{=} \PY{l+m+mi}{1}\PY{p}{;}
        \PY{c}{\PYZpc{} se cosntruyen dos estructuras para almacenar la dominancia y frente al}
        \PY{c}{\PYZpc{} que pertenecen los individuos}
        \PY{n}{F}\PY{p}{(}\PY{n}{frente}\PY{p}{)}\PY{p}{.}\PY{n}{f} \PY{p}{=} \PY{p}{[}\PY{p}{]}\PY{p}{;}
        \PY{n}{individuo} \PY{p}{=} \PY{p}{[}\PY{p}{]}\PY{p}{;}
        \PY{c+cm}{\PYZpc{}\PYZob{}}
        \PY{c+cm}{Debido a su extensión, el código se divide en dos secciones relacionadas}
        \PY{c+cm}{con la formación del frente de Pareto y el cálculo de la distancia de apilamiento}
        \PY{c+cm}{\PYZpc{}\PYZcb{}}
\end{Verbatim}


    \hypertarget{formaciuxf3n-del-frente-de-pareto}{%
\subsubsection{Formación del frente de
Pareto}\label{formaciuxf3n-del-frente-de-pareto}}

Para cada par de individuos (incluyendo el individuo en sí), se realiza
una comparación de sus objetivos, con el propósito de determinar la
dominancia.

    \begin{Verbatim}[commandchars=\\\{\}]
{\color{incolor}In [{\color{incolor} }]:} \PY{c}{\PYZpc{} Para cada individuo i}
        \PY{k}{for} \PY{n+nb}{i} \PY{p}{=} \PY{l+m+mi}{1} \PY{p}{:} \PY{n}{N}
        \PY{c}{\PYZpc{} Número de individuos que dominan a este individuo}
        \PY{n}{individuo}\PY{p}{(}\PY{n+nb}{i}\PY{p}{)}\PY{p}{.}\PY{n}{n} \PY{p}{=} \PY{l+m+mi}{0}\PY{p}{;}
        \PY{c}{\PYZpc{} Número de individuos que domina este individuo}
        \PY{n}{individuo}\PY{p}{(}\PY{n+nb}{i}\PY{p}{)}\PY{p}{.}\PY{n}{p} \PY{p}{=} \PY{p}{[}\PY{p}{]}\PY{p}{;}
        \PY{c}{\PYZpc{}para cada individuo j}
        \PY{k}{for} \PY{n+nb}{j} \PY{p}{=} \PY{l+m+mi}{1} \PY{p}{:} \PY{n}{N}
        \PY{c}{\PYZpc{}contadores de cominancia}
        \PY{n}{domina\PYZus{}menos} \PY{p}{=} \PY{l+m+mi}{0}\PY{p}{;}
        \PY{n}{domina\PYZus{}igual} \PY{p}{=} \PY{l+m+mi}{0}\PY{p}{;}
        \PY{n}{domina\PYZus{}mas} \PY{p}{=} \PY{l+m+mi}{0}\PY{p}{;}
        \PY{c}{\PYZpc{}para cada objetivo}
        \PY{k}{for} \PY{n}{k} \PY{p}{=} \PY{l+m+mi}{1} \PY{p}{:} \PY{n}{M}
        \PY{c}{\PYZpc{}se compara el valor del objetivo k, entre el dindividuo i e}
        \PY{c}{\PYZpc{}individuo j}
        \PY{k}{if} \PY{p}{(}\PY{n}{x}\PY{p}{(}\PY{n+nb}{i}\PY{p}{,} \PY{n}{k}\PY{p}{)} \PY{o}{\PYZlt{}} \PY{n}{x}\PY{p}{(}\PY{n+nb}{j}\PY{p}{,} \PY{n}{k}\PY{p}{)}\PY{p}{)}
        \PY{n}{domina\PYZus{}menos} \PY{p}{=} \PY{n}{domina\PYZus{}menos} \PY{o}{+} \PY{l+m+mi}{1}\PY{p}{;}
        \PY{k}{elseif} \PY{p}{(}\PY{n}{x}\PY{p}{(}\PY{n+nb}{i}\PY{p}{,} \PY{n}{k}\PY{p}{)} \PY{o}{==} \PY{n}{x}\PY{p}{(}\PY{n+nb}{j}\PY{p}{,} \PY{n}{k}\PY{p}{)}\PY{p}{)}
        \PY{n}{domina\PYZus{}igual} \PY{p}{=} \PY{n}{domina\PYZus{}igual} \PY{o}{+} \PY{l+m+mi}{1}\PY{p}{;}
        \PY{k}{else}
        \PY{n}{domina\PYZus{}mas} \PY{p}{=} \PY{n}{domina\PYZus{}mas} \PY{o}{+} \PY{l+m+mi}{1}\PY{p}{;}
        \PY{k}{end}
        \PY{k}{end}
        \PY{c}{\PYZpc{}se determina la dominancia entre cada par de individuos}
        \PY{k}{if} \PY{n}{domina\PYZus{}menos} \PY{o}{==} \PY{l+m+mi}{0} \PY{o}{\PYZam{}\PYZam{}} \PY{n}{domina\PYZus{}igual} \PY{o}{\PYZti{}=} \PY{n}{M}
        \PY{n}{individuo}\PY{p}{(}\PY{n+nb}{i}\PY{p}{)}\PY{p}{.}\PY{n}{n} \PY{p}{=} \PY{n}{individuo}\PY{p}{(}\PY{n+nb}{i}\PY{p}{)}\PY{p}{.}\PY{n}{n} \PY{o}{+} \PY{l+m+mi}{1}\PY{p}{;}
        \PY{k}{elseif} \PY{n}{domina\PYZus{}mas} \PY{o}{==} \PY{l+m+mi}{0} \PY{o}{\PYZam{}\PYZam{}} \PY{n}{domina\PYZus{}igual} \PY{o}{\PYZti{}=} \PY{n}{M}
        \PY{n}{individuo}\PY{p}{(}\PY{n+nb}{i}\PY{p}{)}\PY{p}{.}\PY{n}{p} \PY{p}{=} \PY{p}{[}\PY{n}{individuo}\PY{p}{(}\PY{n+nb}{i}\PY{p}{)}\PY{p}{.}\PY{n}{p} \PY{n+nb}{j}\PY{p}{]}\PY{p}{;}
        \PY{k}{end}
        \PY{k}{end}
        \PY{c}{\PYZpc{}si el individuo i no es dominado, se le asigna el individuo i al}
        \PY{c}{\PYZpc{}frente f}
        \PY{k}{if} \PY{n}{individuo}\PY{p}{(}\PY{n+nb}{i}\PY{p}{)}\PY{p}{.}\PY{n}{n} \PY{o}{==} \PY{l+m+mi}{0}
        \PY{n}{x}\PY{p}{(}\PY{n+nb}{i}\PY{p}{,}\PY{n}{M} \PY{o}{+}  \PY{l+m+mi}{1}\PY{p}{)} \PY{p}{=} \PY{l+m+mi}{1}\PY{p}{;}
        \PY{n}{F}\PY{p}{(}\PY{n}{frente}\PY{p}{)}\PY{p}{.}\PY{n}{f} \PY{p}{=} \PY{p}{[}\PY{n}{F}\PY{p}{(}\PY{n}{frente}\PY{p}{)}\PY{p}{.}\PY{n}{f} \PY{n+nb}{i}\PY{p}{]}\PY{p}{;}
        \PY{k}{end}
        \PY{k}{end}
        \PY{c}{\PYZpc{} Una vez formado los frentes mediante la identificación de dominancia, son}
        \PY{c}{\PYZpc{} organizados (categorizados) los individuos de cada frente, para }
        \PY{c}{\PYZpc{} posteriormente calcular la distancia de apilamiento:}
        \PY{c}{\PYZpc{} mientras existan individuos en cada frente}
        \PY{k}{while} \PY{o}{\PYZti{}}\PY{n+nb}{isempty}\PY{p}{(}\PY{n}{F}\PY{p}{(}\PY{n}{frente}\PY{p}{)}\PY{p}{.}\PY{n}{f}\PY{p}{)}
        \PY{c}{\PYZpc{}     se crea una variable auxiliar para almacenar a lso individuos}
        \PY{n}{Q} \PY{p}{=} \PY{p}{[}\PY{p}{]}\PY{p}{;}
        \PY{c}{\PYZpc{}     se realiza un recorrido por todos los miembros del frente}
        \PY{k}{for} \PY{n+nb}{i} \PY{p}{=} \PY{l+m+mi}{1} \PY{p}{:} \PY{n+nb}{length}\PY{p}{(}\PY{n}{F}\PY{p}{(}\PY{n}{frente}\PY{p}{)}\PY{p}{.}\PY{n}{f}\PY{p}{)}
        \PY{c}{\PYZpc{}si el individuo domina a otro}
        \PY{k}{if} \PY{o}{\PYZti{}}\PY{n+nb}{isempty}\PY{p}{(}\PY{n}{individuo}\PY{p}{(}\PY{n}{F}\PY{p}{(}\PY{n}{frente}\PY{p}{)}\PY{p}{.}\PY{n}{f}\PY{p}{(}\PY{n+nb}{i}\PY{p}{)}\PY{p}{)}\PY{p}{.}\PY{n}{p}\PY{p}{)}
        \PY{c}{\PYZpc{}             se compara la dominancia de lso individuos en el frente}
        \PY{k}{for} \PY{n+nb}{j} \PY{p}{=} \PY{l+m+mi}{1} \PY{p}{:} \PY{n+nb}{length}\PY{p}{(}\PY{n}{individuo}\PY{p}{(}\PY{n}{F}\PY{p}{(}\PY{n}{frente}\PY{p}{)}\PY{p}{.}\PY{n}{f}\PY{p}{(}\PY{n+nb}{i}\PY{p}{)}\PY{p}{)}\PY{p}{.}\PY{n}{p}\PY{p}{)}
        \PY{n}{individuo}\PY{p}{(}\PY{n}{individuo}\PY{p}{(}\PY{n}{F}\PY{p}{(}\PY{n}{frente}\PY{p}{)}\PY{p}{.}\PY{n}{f}\PY{p}{(}\PY{n+nb}{i}\PY{p}{)}\PY{p}{)}\PY{p}{.}\PY{n}{p}\PY{p}{(}\PY{n+nb}{j}\PY{p}{)}\PY{p}{)}\PY{p}{.}\PY{n}{n} \PY{p}{=} \PY{c}{...}
        \PY{n}{individuo}\PY{p}{(}\PY{n}{individuo}\PY{p}{(}\PY{n}{F}\PY{p}{(}\PY{n}{frente}\PY{p}{)}\PY{p}{.}\PY{n}{f}\PY{p}{(}\PY{n+nb}{i}\PY{p}{)}\PY{p}{)}\PY{p}{.}\PY{n}{p}\PY{p}{(}\PY{n+nb}{j}\PY{p}{)}\PY{p}{)}\PY{p}{.}\PY{n}{n} \PY{o}{\PYZhy{}} \PY{l+m+mi}{1}\PY{p}{;}
        \PY{c}{\PYZpc{}se actualzian los valores de dominancia para los}
        \PY{c}{\PYZpc{}individuos del mismo frente y se organizan en la variable}
        \PY{c}{\PYZpc{}auxiliar Q}
        \PY{k}{if} \PY{n}{individuo}\PY{p}{(}\PY{n}{individuo}\PY{p}{(}\PY{n}{F}\PY{p}{(}\PY{n}{frente}\PY{p}{)}\PY{p}{.}\PY{n}{f}\PY{p}{(}\PY{n+nb}{i}\PY{p}{)}\PY{p}{)}\PY{p}{.}\PY{n}{p}\PY{p}{(}\PY{n+nb}{j}\PY{p}{)}\PY{p}{)}\PY{p}{.}\PY{n}{n} \PY{o}{==} \PY{l+m+mi}{0}
        \PY{n}{x}\PY{p}{(}\PY{n}{individuo}\PY{p}{(}\PY{n}{F}\PY{p}{(}\PY{n}{frente}\PY{p}{)}\PY{p}{.}\PY{n}{f}\PY{p}{(}\PY{n+nb}{i}\PY{p}{)}\PY{p}{)}\PY{p}{.}\PY{n}{p}\PY{p}{(}\PY{n+nb}{j}\PY{p}{)}\PY{p}{,}\PY{n}{M} \PY{o}{+}  \PY{l+m+mi}{1}\PY{p}{)} \PY{p}{=} \PY{n}{frente} \PY{o}{+} \PY{l+m+mi}{1}\PY{p}{;}
        \PY{n}{Q} \PY{p}{=} \PY{p}{[}\PY{n}{Q} \PY{n}{individuo}\PY{p}{(}\PY{n}{F}\PY{p}{(}\PY{n}{frente}\PY{p}{)}\PY{p}{.}\PY{n}{f}\PY{p}{(}\PY{n+nb}{i}\PY{p}{)}\PY{p}{)}\PY{p}{.}\PY{n}{p}\PY{p}{(}\PY{n+nb}{j}\PY{p}{)}\PY{p}{]}\PY{p}{;}
        \PY{k}{end}
        \PY{k}{end}
        \PY{k}{end}
        \PY{k}{end}
        \PY{c}{\PYZpc{}se pasa al siguiente frente}
        \PY{n}{frente} \PY{p}{=}  \PY{n}{frente} \PY{o}{+} \PY{l+m+mi}{1}\PY{p}{;}
        \PY{c}{\PYZpc{}Se actualiza el orden de los individuos según el frente auxiliar Q}
        \PY{n}{F}\PY{p}{(}\PY{n}{frente}\PY{p}{)}\PY{p}{.}\PY{n}{f} \PY{p}{=} \PY{n}{Q}\PY{p}{;}
        \PY{k}{end}
        \PY{c}{\PYZpc{} se extrae el orden de los individuos en la varialbe de funciones objetivo}
        \PY{p}{[}\PY{o}{\PYZti{}}\PY{p}{,}\PY{n}{indice\PYZus{}frentes}\PY{p}{]} \PY{p}{=} \PY{n}{sort}\PY{p}{(}\PY{n}{x}\PY{p}{(}\PY{p}{:}\PY{p}{,}\PY{n}{M} \PY{o}{+}  \PY{l+m+mi}{1}\PY{p}{)}\PY{p}{)}\PY{p}{;}
        \PY{c}{\PYZpc{} se organizan los individuso por cada frente}
        \PY{k}{for} \PY{n+nb}{i} \PY{p}{=} \PY{l+m+mi}{1} \PY{p}{:} \PY{n+nb}{length}\PY{p}{(}\PY{n}{indice\PYZus{}frentes}\PY{p}{)}
        \PY{n}{organizado\PYZus{}por\PYZus{}frentes}\PY{p}{(}\PY{n+nb}{i}\PY{p}{,}\PY{p}{:}\PY{p}{)} \PY{p}{=} \PY{n}{x}\PY{p}{(}\PY{n}{indice\PYZus{}frentes}\PY{p}{(}\PY{n+nb}{i}\PY{p}{)}\PY{p}{,}\PY{p}{:}\PY{p}{)}\PY{p}{;}
        \PY{k}{end}
\end{Verbatim}


    \hypertarget{cuxe1lculo-de-la-distancia-de-apilamiento}{%
\subsubsection{Cálculo de la distancia de
apilamiento}\label{cuxe1lculo-de-la-distancia-de-apilamiento}}

Se calcula la distancia de apilamiento, para ello se evalúa cada frente
teniendo en cuenta las funciones objetivo, reorganizando la población en
cada frente para asignarle a los valores extremos (máximo y mínimo de
cada individuo en la función objetivo el valor de infinito),
posteriormente se organizan los individuos según su distancia.

    \begin{Verbatim}[commandchars=\\\{\}]
{\color{incolor}In [{\color{incolor} }]:} \PY{c}{\PYZpc{} se crea un contador}
        \PY{n}{indice\PYZus{}actual} \PY{p}{=} \PY{l+m+mi}{0}\PY{p}{;}
        \PY{c}{\PYZpc{} para todos los frentes menos el último (el cual está vacio por defecto)}
        \PY{k}{for} \PY{n}{frente} \PY{p}{=} \PY{l+m+mi}{1} \PY{p}{:} \PY{p}{(}\PY{n+nb}{length}\PY{p}{(}\PY{n}{F}\PY{p}{)} \PY{o}{\PYZhy{}} \PY{l+m+mi}{1}\PY{p}{)}
        \PY{c}{\PYZpc{}se crea una variable de distancia}
        \PY{n}{distancia} \PY{p}{=} \PY{l+m+mi}{0}\PY{p}{;}
        \PY{c}{\PYZpc{}se crea una variable auxiliar para almacenar la distancia de}
        \PY{c}{\PYZpc{}apilamiento en cada frente}
        \PY{n}{y} \PY{p}{=} \PY{p}{[}\PY{p}{]}\PY{p}{;}
        \PY{c}{\PYZpc{}     se crean dos contadores de índices para comparar los individuos}
        \PY{n}{indice\PYZus{}previo} \PY{p}{=} \PY{n}{indice\PYZus{}actual} \PY{o}{+} \PY{l+m+mi}{1}\PY{p}{;}
        \PY{c}{\PYZpc{}para todos los individuos que comparten frente}
        \PY{k}{for} \PY{n+nb}{i} \PY{p}{=} \PY{l+m+mi}{1} \PY{p}{:} \PY{n+nb}{length}\PY{p}{(}\PY{n}{F}\PY{p}{(}\PY{n}{frente}\PY{p}{)}\PY{p}{.}\PY{n}{f}\PY{p}{)}
        \PY{c}{\PYZpc{}se actualizan los individuos en la variable auziliar}
        \PY{n}{y}\PY{p}{(}\PY{n+nb}{i}\PY{p}{,}\PY{p}{:}\PY{p}{)} \PY{p}{=} \PY{n}{organizado\PYZus{}por\PYZus{}frentes}\PY{p}{(}\PY{n}{indice\PYZus{}actual} \PY{o}{+} \PY{n+nb}{i}\PY{p}{,}\PY{p}{:}\PY{p}{)}\PY{p}{;}
        \PY{k}{end}
        \PY{c}{\PYZpc{}se actualizan los contadores}
        \PY{n}{indice\PYZus{}actual} \PY{p}{=} \PY{n}{indice\PYZus{}actual} \PY{o}{+} \PY{n+nb}{i}\PY{p}{;}
        \PY{c}{\PYZpc{}como se van a organizar por cada función objetivo, se crea una}
        \PY{c}{\PYZpc{}variable auxiliar}
        \PY{n}{organizado\PYZus{}por\PYZus{}objetivos} \PY{p}{=} \PY{p}{[}\PY{p}{]}\PY{p}{;}
        \PY{c}{\PYZpc{}se recorren todos los objetivos}
        \PY{k}{for} \PY{n+nb}{i} \PY{p}{=} \PY{l+m+mi}{1} \PY{p}{:} \PY{n}{M}
        \PY{c}{\PYZpc{}se extrae la organización de los individuos que pertenecen a cada}
        \PY{c}{\PYZpc{} frente, según su valor en la función objetivo M}
        \PY{p}{[}\PY{n}{organizado\PYZus{}por\PYZus{}objetivos}\PY{p}{,} \PY{n}{indice\PYZus{}de\PYZus{}objetivos}\PY{p}{]} \PY{p}{=} \PY{n}{sort}\PY{p}{(}\PY{n}{y}\PY{p}{(}\PY{p}{:}\PY{p}{,} \PY{n+nb}{i}\PY{p}{)}\PY{p}{)}\PY{p}{;}
        \PY{c}{\PYZpc{}         se borra el contenido de la variable auxiliar}
        \PY{n}{organizado\PYZus{}por\PYZus{}objetivos} \PY{p}{=} \PY{p}{[}\PY{p}{]}\PY{p}{;}
        \PY{c}{\PYZpc{}para todos los individuos que se están comparando}
        \PY{k}{for} \PY{n+nb}{j} \PY{p}{=} \PY{l+m+mi}{1} \PY{p}{:} \PY{n+nb}{length}\PY{p}{(}\PY{n}{indice\PYZus{}de\PYZus{}objetivos}\PY{p}{)}
        \PY{c}{\PYZpc{}             se almacena el orden}
        \PY{n}{organizado\PYZus{}por\PYZus{}objetivos}\PY{p}{(}\PY{n+nb}{j}\PY{p}{,}\PY{p}{:}\PY{p}{)} \PY{p}{=} \PY{n}{y}\PY{p}{(}\PY{n}{indice\PYZus{}de\PYZus{}objetivos}\PY{p}{(}\PY{n+nb}{j}\PY{p}{)}\PY{p}{,}\PY{p}{:}\PY{p}{)}\PY{p}{;}
        \PY{k}{end}
        \PY{c}{\PYZpc{}se calcula el valor máximo de la función objetivo obtenido por}
        \PY{c}{\PYZpc{}algún individuo en el frente bajo estudio}
        \PY{n}{f\PYZus{}max} \PY{p}{=} \PY{n}{organizado\PYZus{}por\PYZus{}objetivos}\PY{p}{(}\PY{n+nb}{length}\PY{p}{(}\PY{n}{indice\PYZus{}de\PYZus{}objetivos}\PY{p}{)}\PY{p}{,}  \PY{n+nb}{i}\PY{p}{)}\PY{p}{;}
        \PY{c}{\PYZpc{}se calcula el valor mínimo de la función objetivo obtenido por}
        \PY{c}{\PYZpc{}algún individuo en el frente bajo estudio}
        \PY{n}{f\PYZus{}min} \PY{p}{=} \PY{n}{organizado\PYZus{}por\PYZus{}objetivos}\PY{p}{(}\PY{l+m+mi}{1}\PY{p}{,}  \PY{n+nb}{i}\PY{p}{)}\PY{p}{;}
        \PY{c}{\PYZpc{}se iguala a infinito los valores de distancia de los individuos}
        \PY{c}{\PYZpc{}extremos}
        \PY{n}{y}\PY{p}{(}\PY{n}{indice\PYZus{}de\PYZus{}objetivos}\PY{p}{(}\PY{n+nb}{length}\PY{p}{(}\PY{n}{indice\PYZus{}de\PYZus{}objetivos}\PY{p}{)}\PY{p}{)}\PY{p}{,}\PY{n}{M} \PY{o}{+}  \PY{l+m+mi}{1} \PY{o}{+} \PY{n+nb}{i}\PY{p}{)}\PY{p}{=} \PY{n}{Inf}\PY{p}{;}
        \PY{n}{y}\PY{p}{(}\PY{n}{indice\PYZus{}de\PYZus{}objetivos}\PY{p}{(}\PY{l+m+mi}{1}\PY{p}{)}\PY{p}{,}\PY{n}{M} \PY{o}{+}  \PY{l+m+mi}{1} \PY{o}{+} \PY{n+nb}{i}\PY{p}{)} \PY{p}{=} \PY{n}{Inf}\PY{p}{;}
        \PY{c}{\PYZpc{}se calcula la distancia para todos los individuos que estén en el}
        \PY{c}{\PYZpc{}frente, menos aquellos que pertenecen a los extremos}
        \PY{k}{for} \PY{n+nb}{j} \PY{p}{=} \PY{l+m+mi}{2} \PY{p}{:} \PY{n+nb}{length}\PY{p}{(}\PY{n}{indice\PYZus{}de\PYZus{}objetivos}\PY{p}{)} \PY{o}{\PYZhy{}} \PY{l+m+mi}{1}
        \PY{n}{objetivo\PYZus{}siguiente}  \PY{p}{=} \PY{n}{organizado\PYZus{}por\PYZus{}objetivos}\PY{p}{(}\PY{n+nb}{j} \PY{o}{+} \PY{l+m+mi}{1}\PY{p}{,} \PY{n+nb}{i}\PY{p}{)}\PY{p}{;}
        \PY{n}{objetivo\PYZus{}previo}  \PY{p}{=} \PY{n}{organizado\PYZus{}por\PYZus{}objetivos}\PY{p}{(}\PY{n+nb}{j} \PY{o}{\PYZhy{}} \PY{l+m+mi}{1}\PY{p}{,} \PY{n+nb}{i}\PY{p}{)}\PY{p}{;}
        \PY{c}{\PYZpc{}si los valores de la función objetivo es igual a cero, se le}
        \PY{c}{\PYZpc{}asigna una distancia igual a infinito}
        \PY{k}{if} \PY{p}{(}\PY{n}{f\PYZus{}max} \PY{o}{\PYZhy{}} \PY{n}{f\PYZus{}min} \PY{o}{==} \PY{l+m+mi}{0}\PY{p}{)}
        \PY{n}{y}\PY{p}{(}\PY{n}{indice\PYZus{}de\PYZus{}objetivos}\PY{p}{(}\PY{n+nb}{j}\PY{p}{)}\PY{p}{,}\PY{n}{M} \PY{o}{+}  \PY{l+m+mi}{1} \PY{o}{+} \PY{n+nb}{i}\PY{p}{)} \PY{p}{=} \PY{n}{Inf}\PY{p}{;}
        \PY{c}{\PYZpc{}si el valor de la función objetivo es diferente, se calcula}
        \PY{c}{\PYZpc{}la distancia de apilamiento}
        \PY{k}{else}
        \PY{n}{y}\PY{p}{(}\PY{n}{indice\PYZus{}de\PYZus{}objetivos}\PY{p}{(}\PY{n+nb}{j}\PY{p}{)}\PY{p}{,}\PY{n}{M} \PY{o}{+}  \PY{l+m+mi}{1} \PY{o}{+} \PY{n+nb}{i}\PY{p}{)} \PY{p}{=} \PY{c}{...}
        \PY{p}{(}\PY{n}{objetivo\PYZus{}siguiente} \PY{o}{\PYZhy{}} \PY{n}{objetivo\PYZus{}previo}\PY{p}{)}\PY{o}{/}\PY{p}{(}\PY{n}{f\PYZus{}max} \PY{o}{\PYZhy{}} \PY{n}{f\PYZus{}min}\PY{p}{)}\PY{p}{;}
        \PY{k}{end}
        \PY{k}{end}
        \PY{k}{end}
        \PY{c}{\PYZpc{}se reinicia el valor de distancia}
        \PY{n}{distancia} \PY{p}{=} \PY{p}{[}\PY{p}{]}\PY{p}{;}
        \PY{c}{\PYZpc{}se asigna dimensión a la variable distancia, según la cantidad de}
        \PY{c}{\PYZpc{}individuos en el frente}
        \PY{n}{distancia}\PY{p}{(}\PY{p}{:}\PY{p}{,}\PY{l+m+mi}{1}\PY{p}{)} \PY{p}{=} \PY{n+nb}{zeros}\PY{p}{(}\PY{n+nb}{length}\PY{p}{(}\PY{n}{F}\PY{p}{(}\PY{n}{frente}\PY{p}{)}\PY{p}{.}\PY{n}{f}\PY{p}{)}\PY{p}{,}\PY{l+m+mi}{1}\PY{p}{)}\PY{p}{;}
        \PY{c}{\PYZpc{}se le asigna el valor de distancia según cada objetivo al vector}
        \PY{c}{\PYZpc{}creado}
        \PY{k}{for} \PY{n+nb}{i} \PY{p}{=} \PY{l+m+mi}{1} \PY{p}{:} \PY{n}{M}
        \PY{n}{distancia}\PY{p}{(}\PY{p}{:}\PY{p}{,}\PY{l+m+mi}{1}\PY{p}{)} \PY{p}{=} \PY{n}{distancia}\PY{p}{(}\PY{p}{:}\PY{p}{,}\PY{l+m+mi}{1}\PY{p}{)} \PY{o}{+} \PY{n}{y}\PY{p}{(}\PY{p}{:}\PY{p}{,}\PY{n}{M} \PY{o}{+}  \PY{l+m+mi}{1} \PY{o}{+} \PY{n+nb}{i}\PY{p}{)}\PY{p}{;}
        \PY{k}{end}
        \PY{c}{\PYZpc{}se actualiza en la variable y el valor de la distancia}
        \PY{n}{y}\PY{p}{(}\PY{p}{:}\PY{p}{,}\PY{n}{M} \PY{o}{+}  \PY{l+m+mi}{2}\PY{p}{)} \PY{p}{=} \PY{n}{distancia}\PY{p}{;}
        \PY{n}{y} \PY{p}{=} \PY{n}{y}\PY{p}{(}\PY{p}{:}\PY{p}{,} \PY{n}{M} \PY{o}{+}  \PY{l+m+mi}{2}\PY{p}{)}\PY{p}{;}
        \PY{c}{\PYZpc{}se almacena el valor de distancia en la variable z, la cual contempla}
        \PY{c}{\PYZpc{}todos los frentes}
        \PY{n}{z}\PY{p}{(}\PY{n}{indice\PYZus{}previo}\PY{p}{:}\PY{n}{indice\PYZus{}actual}\PY{p}{,}\PY{p}{:}\PY{p}{)} \PY{p}{=} \PY{n}{y}\PY{p}{;}
        \PY{k}{end}
        \PY{n}{OP\PYZus{}criterio\PYZus{}evaluacion}\PY{p}{(}\PY{l+m+mi}{1}\PY{p}{,}\PY{p}{:}\PY{p}{)}\PY{p}{=}\PY{n}{x}\PY{p}{(}\PY{p}{:}\PY{p}{,}\PY{l+m+mi}{3}\PY{p}{)}\PY{o}{\PYZsq{}}\PY{p}{;}
        \PY{n}{OP\PYZus{}criterio\PYZus{}evaluacion}\PY{p}{(}\PY{l+m+mi}{2}\PY{p}{,}\PY{p}{:}\PY{p}{)}\PY{p}{=}\PY{n}{z}\PY{p}{(}\PY{p}{)}\PY{o}{\PYZsq{}}\PY{p}{;}
        \PY{c}{\PYZpc{}===========================================================================}
        \PY{k}{end}
\end{Verbatim}


    \hypertarget{algortimo-para-seleccionar-el-conjunto-de-padres-seleccion_por_torneo.m}{%
\subsection{Algortimo para seleccionar el conjunto de Padres
(seleccion\_por\_torneo.m)}\label{algortimo-para-seleccionar-el-conjunto-de-padres-seleccion_por_torneo.m}}

La selección de los padres de hace de manera aleatoria a partir del
conjunto total de soluciones (variable denominada `Cromosomas'), cada
par de padres se enfrenta teniendo en cuenta dos parámetros: 1) el valor
del frente a cual pertenece cada individuo, seleccionando el individuo
de menor frente y, 2) el valor de la distancia de apilamiento, en caso
de pertenecer al mismo frente, como criterio de selección está el
individuo con mayor valor de distancia.

las variables de entrada son ST\_Cromosomas (Soluciones a competir),
ST\_Matriz\_objetivos (Funciones de ajuste), ST\_criterio\_evaluacion
(la primera fila corresponde al frente al cual pertenece cada solución y
la segunda fila la distancia de apilamiento de la solución en su
respectivo frente), ST\_pool (cantidad de grupos a enfrentarse) y
ST\_torneo (cantidad de contrincantes). Como variables de salida la
función genera: ST\_Cromosomas\_padres (Soluciones seleccionadas como
ganadoras del torneo), ST\_criterio\_evaluacion\_padres (los respectivos
valores de frente y distancia de apilamiento de las soluciones
seleccionadas) y ST\_Matriz\_objetivos\_padres (matriz de soluciones de
ajuste).

    \begin{Verbatim}[commandchars=\\\{\}]
{\color{incolor}In [{\color{incolor} }]:} \PY{k}{function}\PY{err}{ }\PY{err}{[}\PY{n+nf}{ST\PYZus{}Cromosomas\PYZus{}padres}\PY{p}{,}\PY{n}{ST\PYZus{}criterio\PYZus{}evaluacion\PYZus{}padres}\PY{p}{,}\PY{c}{...}
        \PY{n}{ST\PYZus{}Matriz\PYZus{}objetivos\PYZus{}padres}\PY{p}{]} \PY{p}{=} \PY{n}{seleccion\PYZus{}por\PYZus{}torneo}\PY{p}{(} \PY{n}{ST\PYZus{}Cromosomas}\PY{p}{,} \PY{c}{...}
        \PY{n}{ST\PYZus{}Matriz\PYZus{}objetivos}\PY{p}{,} \PY{n}{ST\PYZus{}criterio\PYZus{}evaluacion}\PY{p}{,} \PY{n}{ST\PYZus{}pool}\PY{p}{,} \PY{n}{ST\PYZus{}torneo} \PY{p}{)}
        \PY{c}{\PYZpc{} Se calcula la cantidad de cromosomas a evaluar}
        \PY{p}{[}\PY{o}{\PYZti{}}\PY{p}{,} \PY{n}{cant\PYZus{}cromosomas}\PY{p}{]}\PY{p}{=}\PY{n+nb}{size}\PY{p}{(}\PY{n}{ST\PYZus{}Cromosomas}\PY{p}{)}\PY{p}{;}
        \PY{c}{\PYZpc{} Se calcula la cantidad de individuos}
        \PY{p}{[}\PY{n}{variables}\PY{p}{,} \PY{n}{poblacion}\PY{p}{]} \PY{p}{=} \PY{n+nb}{size}\PY{p}{(}\PY{n}{ST\PYZus{}criterio\PYZus{}evaluacion}\PY{p}{)}\PY{p}{;}
        \PY{c}{\PYZpc{} Se almacena la cantidad de individuos en una variable auxiliar}
        \PY{n}{Poblacion\PYZus{}total}\PY{p}{=}\PY{n}{poblacion}\PY{p}{;}
        \PY{c}{\PYZpc{} Se computa la cantidad de lotes del problema}
        \PY{n}{cant\PYZus{}lotes}\PY{p}{=}\PY{n}{cant\PYZus{}cromosomas}\PY{o}{/}\PY{n}{poblacion}\PY{p}{;}
        \PY{c}{\PYZpc{} Se crea un vector auxiliar para determinar cuáles miembros son seleccionados:}
        \PY{n}{poblacion}\PY{p}{=}\PY{l+m+mi}{1}\PY{p}{:}\PY{n}{poblacion}\PY{p}{;}
        \PY{c}{\PYZpc{} Se almacena el frente al que pertenece cada solución en la variable rango}
        \PY{n}{rango} \PY{p}{=} \PY{n}{ST\PYZus{}criterio\PYZus{}evaluacion}\PY{p}{(}\PY{n}{variables} \PY{o}{\PYZhy{}} \PY{l+m+mi}{1}\PY{p}{,}\PY{p}{:}\PY{p}{)}\PY{p}{;}
        \PY{c}{\PYZpc{} Se almacena el valor de distancia de cada solución en la variable}
        \PY{c}{\PYZpc{} distancia}
        \PY{n}{distancia} \PY{p}{=} \PY{n}{ST\PYZus{}criterio\PYZus{}evaluacion}\PY{p}{(}\PY{n}{variables}\PY{p}{,}\PY{p}{:}\PY{p}{)}\PY{p}{;}
        \PY{c}{\PYZpc{} Se genera un contador con todos la cantidad de combates o enfrentamientos}
        \PY{k}{for} \PY{n+nb}{i} \PY{p}{=} \PY{l+m+mi}{1} \PY{p}{:} \PY{n}{ST\PYZus{}pool}
        \PY{c+cm}{\PYZpc{}\PYZob{}}
        \PY{c+cm}{Como a medida que pasan los enfrentamientos la cantidad de individuos}
        \PY{c+cm}{disminuye, se crea una condición de selección, la cual sólo aplica luego}
        \PY{c+cm}{del primer enfrentamiento.}
        \PY{c+cm}{para el primer enfrentamiento}
        \PY{c+cm}{\PYZpc{}\PYZcb{}}
        \PY{k}{if} \PY{n+nb}{i}\PY{o}{==}\PY{l+m+mi}{1}
        \PY{c}{\PYZpc{}seleccione los contrincantes}
        \PY{n}{combatientes}\PY{p}{=}\PY{n}{datasample}\PY{p}{(}\PY{n}{poblacion}\PY{p}{,}\PY{n}{ST\PYZus{}torneo}\PY{p}{,}\PY{l+s}{\PYZsq{}}\PY{l+s}{Replace\PYZsq{}}\PY{p}{,}\PY{n}{false}\PY{p}{)}\PY{p}{;}
        \PY{c}{\PYZpc{}         almacene el valor del frente}
        \PY{n}{rango}\PY{p}{=}\PY{n}{ST\PYZus{}criterio\PYZus{}evaluacion}\PY{p}{(}\PY{l+m+mi}{1}\PY{p}{,}\PY{n}{combatientes}\PY{p}{)}\PY{p}{;}
        \PY{c}{\PYZpc{}almacene el valor de la distancia}
        \PY{n}{distancia}\PY{p}{=}\PY{n}{ST\PYZus{}criterio\PYZus{}evaluacion}\PY{p}{(}\PY{l+m+mi}{2}\PY{p}{,}\PY{n}{combatientes}\PY{p}{)}\PY{p}{;}
        \PY{c}{\PYZpc{} Para el resto de enfrentamientos}
        \PY{k}{else}
        \PY{c}{\PYZpc{}Seleccione los contrincantes que no han sido seleccionado}
        \PY{c}{\PYZpc{}anteriormente}
        \PY{n}{combatientes}\PY{p}{=}\PY{n}{datasample}\PY{p}{(}\PY{n}{poblacion}\PY{p}{(}\PY{n}{poblacion}\PY{o}{\PYZti{}=}\PY{l+m+mi}{0}\PY{p}{)}\PY{p}{,}\PY{n}{ST\PYZus{}torneo}\PY{p}{,}\PY{l+s}{\PYZsq{}}\PY{l+s}{Replace\PYZsq{}}\PY{p}{,}\PY{n}{false}\PY{p}{)}\PY{p}{;}
        \PY{c}{\PYZpc{}almacene el valor del frente}
        \PY{n}{rango}\PY{p}{=}\PY{n}{ST\PYZus{}criterio\PYZus{}evaluacion}\PY{p}{(}\PY{l+m+mi}{1}\PY{p}{,}\PY{n}{combatientes}\PY{p}{)}\PY{p}{;}
        \PY{c}{\PYZpc{}almacene el valor de la distancia}
        \PY{n}{distancia}\PY{p}{=}\PY{n}{ST\PYZus{}criterio\PYZus{}evaluacion}\PY{p}{(}\PY{l+m+mi}{2}\PY{p}{,}\PY{n}{combatientes}\PY{p}{)}\PY{p}{;}
        \PY{k}{end}
        \PY{c}{\PYZpc{} Seleccione el candidato con el menor rango}
        \PY{n}{min\PYZus{}candidato} \PY{p}{=} \PY{n+nb}{find}\PY{p}{(}\PY{n}{rango} \PY{o}{==} \PY{n}{min}\PY{p}{(}\PY{n}{rango}\PY{p}{)}\PY{p}{)}\PY{p}{;}
        \PY{c+cm}{\PYZpc{}\PYZob{}}
        \PY{c+cm}{Como es posible que los dos contrincantes pertenezcan al mismo rango, el}
        \PY{c+cm}{valor de miembros de \PYZsq{}min\PYZus{}candidato\PYZsq{} puede ser mayor de 1, pro tanto:}
        \PY{c+cm}{si hay más de una solución:}
        \PY{c+cm}{\PYZpc{}\PYZcb{}}
        \PY{k}{if} \PY{n+nb}{length}\PY{p}{(}\PY{n}{min\PYZus{}candidato}\PY{p}{)} \PY{o}{\PYZti{}=} \PY{l+m+mi}{1}
        \PY{c}{\PYZpc{}seleccione el candidato con la mayor distancia}
        \PY{n}{max\PYZus{}candidato} \PY{p}{=} \PY{n+nb}{find}\PY{p}{(}\PY{n}{distancia}\PY{p}{(}\PY{n}{min\PYZus{}candidato}\PY{p}{)} \PY{o}{==} \PY{n}{max}\PY{p}{(}\PY{n}{distancia}\PY{p}{(}\PY{n}{min\PYZus{}candidato}\PY{p}{)}\PY{p}{)}\PY{p}{)}\PY{p}{;}
        \PY{c}{\PYZpc{} si los dos sujetos tienen la misma distnacia, seleccione cualquiera (por}
        \PY{c}{\PYZpc{} defecto, el primero)}
        \PY{k}{if} \PY{n+nb}{length}\PY{p}{(}\PY{n}{max\PYZus{}candidato}\PY{p}{)} \PY{o}{\PYZti{}=} \PY{l+m+mi}{1}
        \PY{n}{max\PYZus{}candidato} \PY{p}{=} \PY{n}{max\PYZus{}candidato}\PY{p}{(}\PY{l+m+mi}{1}\PY{p}{)}\PY{p}{;}
        \PY{k}{end}
        \PY{c}{\PYZpc{} Se selecciona el valor de la solución y se almacena el la variable de}
        \PY{c}{\PYZpc{} salida}
        \PY{n}{ST\PYZus{}Matriz\PYZus{}objetivos\PYZus{}padres}\PY{p}{(}\PY{p}{:}\PY{p}{,}\PY{n+nb}{i}\PY{p}{)}\PY{p}{=}\PY{n}{ST\PYZus{}Matriz\PYZus{}objetivos}\PY{p}{(}\PY{p}{:}\PY{p}{,}\PY{c}{...}
        \PY{n}{combatientes}\PY{p}{(}\PY{n}{min\PYZus{}candidato}\PY{p}{(}\PY{n}{max\PYZus{}candidato}\PY{p}{)}\PY{p}{)}\PY{p}{)}\PY{p}{;}
        \PY{c}{\PYZpc{} Se almacenan los correspondientes criteros de evaluación (valor de frente}
        \PY{c}{\PYZpc{} y distancia)}
        \PY{n}{ST\PYZus{}criterio\PYZus{}evaluacion\PYZus{}padres}\PY{p}{(}\PY{p}{:}\PY{p}{,}\PY{n+nb}{i}\PY{p}{)}\PY{p}{=}\PY{n}{ST\PYZus{}criterio\PYZus{}evaluacion}\PY{p}{(}\PY{p}{:}\PY{p}{,}\PY{c}{...}
        \PY{n}{combatientes}\PY{p}{(}\PY{n}{min\PYZus{}candidato}\PY{p}{(}\PY{n}{max\PYZus{}candidato}\PY{p}{)}\PY{p}{)}\PY{p}{)}\PY{p}{;}
        \PY{c}{\PYZpc{} Se almacena la solución en la nueva variable de cromosomas}
        \PY{n}{ST\PYZus{}Cromosomas\PYZus{}padres}\PY{p}{(}\PY{p}{:}\PY{p}{,}\PY{p}{(}\PY{n+nb}{i}\PY{o}{\PYZhy{}}\PY{l+m+mi}{1}\PY{p}{)}\PY{o}{*}\PY{n}{cant\PYZus{}lotes}\PY{o}{+}\PY{l+m+mi}{1}\PY{p}{:}\PY{p}{(}\PY{n+nb}{i}\PY{o}{\PYZhy{}}\PY{l+m+mi}{1}\PY{p}{)}\PY{o}{*}\PY{n}{cant\PYZus{}lotes}\PY{o}{+}\PY{n}{cant\PYZus{}lotes}\PY{p}{)} \PY{p}{=} \PY{c}{...}
        \PY{n}{ST\PYZus{}Cromosomas}\PY{p}{(}\PY{p}{:}\PY{p}{,}\PY{p}{(}\PY{p}{(}\PY{n}{combatientes}\PY{p}{(}\PY{n}{min\PYZus{}candidato}\PY{p}{(}\PY{n}{max\PYZus{}candidato}\PY{p}{)}\PY{p}{)}\PY{p}{)}\PY{o}{\PYZhy{}}\PY{l+m+mi}{1}\PY{p}{)}\PY{o}{*}\PY{c}{...}
        \PY{n}{cant\PYZus{}lotes}\PY{o}{+}\PY{l+m+mi}{1}\PY{p}{:}\PY{p}{(}\PY{p}{(}\PY{n}{combatientes}\PY{p}{(}\PY{n}{min\PYZus{}candidato}\PY{p}{(}\PY{n}{max\PYZus{}candidato}\PY{p}{)}\PY{p}{)}\PY{p}{)}\PY{o}{\PYZhy{}}\PY{l+m+mi}{1}\PY{p}{)}\PY{o}{*}\PY{n}{cant\PYZus{}lotes}\PY{o}{+}\PY{n}{cant\PYZus{}lotes}\PY{p}{)}\PY{p}{;}
        \PY{c}{\PYZpc{}se elimina el ganador del pool}
        \PY{n}{poblacion}\PY{p}{(}\PY{n}{combatientes}\PY{p}{(}\PY{n}{min\PYZus{}candidato}\PY{p}{(}\PY{n}{max\PYZus{}candidato}\PY{p}{)}\PY{p}{)}\PY{p}{)}\PY{p}{=}\PY{l+m+mi}{0}\PY{p}{;}
        \PY{c}{\PYZpc{} Sí sólo existe un único contrincante por frente:}
        \PY{k}{else}
        \PY{c}{\PYZpc{} Se selecciona el valor de la solución y se almacena el la variable de}
        \PY{c}{\PYZpc{} salida}
        \PY{n}{ST\PYZus{}Matriz\PYZus{}objetivos\PYZus{}padres}\PY{p}{(}\PY{p}{:}\PY{p}{,}\PY{n+nb}{i}\PY{p}{)}\PY{p}{=}\PY{n}{ST\PYZus{}Matriz\PYZus{}objetivos}\PY{p}{(}\PY{p}{:}\PY{p}{,}\PY{n}{combatientes}\PY{p}{(}\PY{n}{min\PYZus{}candidato}\PY{p}{(}\PY{l+m+mi}{1}\PY{p}{)}\PY{p}{)}\PY{p}{)}\PY{p}{;}
        \PY{c}{\PYZpc{} Se almacenan los correspondientes criteros de evaluación (valor de frente}
        \PY{c}{\PYZpc{} y distancia)}
        \PY{n}{ST\PYZus{}criterio\PYZus{}evaluacion\PYZus{}padres}\PY{p}{(}\PY{p}{:}\PY{p}{,}\PY{n+nb}{i}\PY{p}{)}\PY{p}{=}\PY{n}{ST\PYZus{}criterio\PYZus{}evaluacion}\PY{p}{(}\PY{p}{:}\PY{p}{,}\PY{c}{...}
        \PY{n}{combatientes}\PY{p}{(}\PY{n}{min\PYZus{}candidato}\PY{p}{(}\PY{l+m+mi}{1}\PY{p}{)}\PY{p}{)}\PY{p}{)}\PY{p}{;}
        \PY{p}{(}\PY{p}{(}\PY{n}{combatientes}\PY{p}{(}\PY{n}{min\PYZus{}candidato}\PY{p}{(}\PY{l+m+mi}{1}\PY{p}{)}\PY{p}{)}\PY{p}{)}\PY{o}{\PYZhy{}}\PY{l+m+mi}{1}\PY{p}{)}\PY{o}{*}\PY{n}{cant\PYZus{}lotes}\PY{o}{+}\PY{l+m+mi}{1}\PY{p}{:}\PY{p}{(}\PY{p}{(}\PY{n}{combatientes}\PY{p}{(}\PY{n}{min\PYZus{}candidato}\PY{p}{(}\PY{l+m+mi}{1}\PY{p}{)}\PY{p}{)}\PY{p}{)}\PY{o}{\PYZhy{}}\PY{l+m+mi}{1}\PY{p}{)}\PY{c}{...}
        \PY{o}{*}\PY{n}{cant\PYZus{}lotes}\PY{o}{+}\PY{n}{cant\PYZus{}lotes}\PY{p}{;}
        \PY{c}{\PYZpc{} Se almacena la solución en la nueva variable de cromosomas}
        \PY{n}{ST\PYZus{}Cromosomas\PYZus{}padres}\PY{p}{(}\PY{p}{:}\PY{p}{,}\PY{p}{(}\PY{n+nb}{i}\PY{o}{\PYZhy{}}\PY{l+m+mi}{1}\PY{p}{)}\PY{o}{*}\PY{n}{cant\PYZus{}lotes}\PY{o}{+}\PY{l+m+mi}{1}\PY{p}{:}\PY{p}{(}\PY{n+nb}{i}\PY{o}{\PYZhy{}}\PY{l+m+mi}{1}\PY{p}{)}\PY{o}{*}\PY{n}{cant\PYZus{}lotes}\PY{o}{+}\PY{n}{cant\PYZus{}lotes}\PY{p}{)} \PY{p}{=} \PY{c}{...}
        \PY{n}{ST\PYZus{}Cromosomas}\PY{p}{(}\PY{p}{:}\PY{p}{,}\PY{p}{(}\PY{p}{(}\PY{n}{combatientes}\PY{p}{(}\PY{n}{min\PYZus{}candidato}\PY{p}{(}\PY{l+m+mi}{1}\PY{p}{)}\PY{p}{)}\PY{p}{)}\PY{o}{\PYZhy{}}\PY{l+m+mi}{1}\PY{p}{)}\PY{o}{*}\PY{n}{cant\PYZus{}lotes}\PY{o}{+}\PY{l+m+mf}{1.}\PY{p}{.}\PY{p}{.}
        \PY{p}{:}\PY{p}{(}\PY{p}{(}\PY{n}{combatientes}\PY{p}{(}\PY{n}{min\PYZus{}candidato}\PY{p}{(}\PY{l+m+mi}{1}\PY{p}{)}\PY{p}{)}\PY{p}{)}\PY{o}{\PYZhy{}}\PY{l+m+mi}{1}\PY{p}{)}\PY{o}{*}\PY{n}{cant\PYZus{}lotes}\PY{o}{+}\PY{n}{cant\PYZus{}lotes}\PY{p}{)}\PY{p}{;}
        \PY{c}{\PYZpc{}         se elimina el ganador del pool}
        \PY{n}{poblacion}\PY{p}{(}\PY{n}{combatientes}\PY{p}{(}\PY{n}{min\PYZus{}candidato}\PY{p}{(}\PY{l+m+mi}{1}\PY{p}{)}\PY{p}{)}\PY{p}{)}\PY{p}{=}\PY{l+m+mi}{0}\PY{p}{;}
        \PY{k}{end}
        \PY{k}{end}
        \PY{k}{end}
\end{Verbatim}


    \hypertarget{algortimo-para-crear-el-conjunto-de-hijos-operador_genetico.m}{%
\subsection{Algortimo para crear el conjunto de Hijos
(operador\_genetico.m)}\label{algortimo-para-crear-el-conjunto-de-hijos-operador_genetico.m}}

La función `operador\_genetico' crea hijos a partir de los padres
seleccionados, para ello existen dos posibildiades: 1) generar dos hijos
a partir del cruce -en un único punto- entre dos padres. 2) la
generación de 2 hijos a partir de la mutación en un segmento del
cromosoma de ambos padres. la función genera la solución y el respectivo
valor de las funciones de ajuste.

Como el existe la posibilidad de generar nuevos segmentos del cromosoma
(cuando se aplica la mutación), es necesario cargar las variables para
generar los cromosomas. las variables de entrada del modelo son:
OG\_cant\_objetivos (cantidad de objetivos a evaluar),
OG\_cant\_periodos (tamaño del horizonte del planeación),
OG\_cant\_lotes (cantidad de lotes a ser cultivados),
OG\_Cromosomas\_padres(conjunto de soluciones ya pre seleccionads)
OG\_probabilidad\_mutacion (Probabilidad de generar el hijo a partir de
la mutación del padre), OG\_poblacion (tamaño de la población),
OG\_cant\_productos (cantidad de productos a cultivar), OG\_Covkkp
(covarianza entre los rendimientos o retornos económicos de cada
producto), OG\_precio\_venta (Precio de venta de cada producto para cada
semana), OG\_rendimiento (cantidad de kilogramos por metro cuadrado a
recoger de cada producto), OG\_areas (tamaño en metros cuadrados de cada
lote), OG\_demanda (Demanda de cada categoría de productos),
OG\_familia\_botanica (Familia botánica a la que pertenece cada
producto), OG\_familia\_venta (Categorías o familia de venta a la cual
pertenece cada producto). y como variables de salida: Cromosomas\_hijos
(conjunto de soluciones generadas a partir de las transformaciones
genéticas) y objetivos\_hijos (valor de las funciones de ajuste para las
soluciones creadas).

Las variables de entrada son: poblacion cant\_objetivos cant\_periodos
cant\_productos cant\_lotes precio\_venta rendimiento areas demanda
familia\_botanica familia\_venta y Covkkp.

    \begin{Verbatim}[commandchars=\\\{\}]
{\color{incolor}In [{\color{incolor} }]:} \PY{k}{function}\PY{err}{ }\PY{err}{[}\PY{n+nf}{Cromosomas\PYZus{}hijos}\PY{p}{,} \PY{n}{objetivos\PYZus{}hijos}\PY{p}{]} \PY{p}{=} \PY{c}{...}
        \PY{n}{operador\PYZus{}genetico}\PY{p}{(} \PY{n}{OG\PYZus{}cant\PYZus{}objetivos}\PY{p}{,} \PY{n}{OG\PYZus{}cant\PYZus{}periodos}\PY{p}{,} \PY{c}{...}
        \PY{n}{OG\PYZus{}cant\PYZus{}lotes}\PY{p}{,}\PY{n}{OG\PYZus{}Cromosomas\PYZus{}padres}\PY{p}{,}\PY{n}{OG\PYZus{}probabilidad\PYZus{}mutacion}\PY{p}{,}\PY{c}{...}
        \PY{n}{OG\PYZus{}poblacion}\PY{p}{,}\PY{n}{OG\PYZus{}cant\PYZus{}productos}\PY{p}{,}\PY{n}{OG\PYZus{}Covkkp}\PY{p}{,}\PY{n}{OG\PYZus{}precio\PYZus{}venta}\PY{p}{,}\PY{c}{...}
        \PY{n}{OG\PYZus{}rendimiento}\PY{p}{,}\PY{n}{OG\PYZus{}areas}\PY{p}{,}\PY{n}{OG\PYZus{}demanda}\PY{p}{,}\PY{n}{OG\PYZus{}familia\PYZus{}botanica}\PY{p}{,}\PY{c}{...}
        \PY{n}{OG\PYZus{}familia\PYZus{}venta}\PY{p}{)}
        \PY{c}{\PYZpc{}Se calcula la cantidad de individuos en la población de padres}
        \PY{p}{[}\PY{o}{\PYZti{}}\PY{p}{,}\PY{n}{N}\PY{p}{]} \PY{p}{=} \PY{n+nb}{size}\PY{p}{(}\PY{n}{OG\PYZus{}Cromosomas\PYZus{}padres}\PY{p}{)}\PY{p}{;}
        \PY{n}{N}\PY{p}{=}\PY{n}{N}\PY{o}{/}\PY{n}{OG\PYZus{}cant\PYZus{}lotes}\PY{p}{;}
        \PY{c}{\PYZpc{}Se genera una variable auxiliar para almacenar el valor de las funciones}
        \PY{c}{\PYZpc{}de ajuste de cada nuevo indidviduo}
        \PY{n}{Matriz\PYZus{}objetivos}\PY{p}{=}\PY{n+nb}{zeros}\PY{p}{(}\PY{n}{OG\PYZus{}cant\PYZus{}objetivos}\PY{o}{+}\PY{l+m+mi}{1}\PY{p}{,}\PY{l+m+mi}{1}\PY{p}{)}\PY{p}{;}
        \PY{c}{\PYZpc{}Se cargan los parámetros}
        \PY{n}{load}\PY{p}{(}\PY{l+s}{\PYZsq{}}\PY{l+s}{parametros\PYZus{}maduracion.mat\PYZsq{}}\PY{p}{)}
        \PY{c+cm}{\PYZpc{}\PYZob{}}
        \PY{c+cm}{PPP = Productos Por Periodo}
        \PY{c+cm}{MTS = Matriz Tiempos Siembra}
        \PY{c+cm}{MFS = Matriz Fecha Siembra}
        \PY{c+cm}{PM  = Periodo de maduración}
        \PY{c+cm}{PMS = Periodo de maduración en semanas}
        \PY{c+cm}{para la prueba trabajaré con 5 sujetos}
        \PY{c+cm}{Se genera un bucle para todos los padres}
        \PY{c+cm}{\PYZpc{}\PYZcb{}}
        \PY{k}{for} \PY{n+nb}{i} \PY{p}{=} \PY{l+m+mi}{1} \PY{p}{:} \PY{n}{N}
        \PY{c}{\PYZpc{} Se determina si se aplica una mutación o cruce}
        \PY{k}{if} \PY{n+nb}{rand}\PY{p}{(}\PY{p}{)} \PY{o}{\PYZlt{}} \PY{l+m+mi}{1}\PY{o}{\PYZhy{}} \PY{n}{OG\PYZus{}probabilidad\PYZus{}mutacion}
        \PY{c}{\PYZpc{} Se crean los hivos (para el caso de cruce)}
        \PY{n}{hijo\PYZus{}1} \PY{p}{=} \PY{p}{[}\PY{p}{]}\PY{p}{;}
        \PY{n}{hijo\PYZus{}2} \PY{p}{=} \PY{p}{[}\PY{p}{]}\PY{p}{;}
        \PY{c}{\PYZpc{} Se seleccionan los padres}
        \PY{n}{padres}\PY{p}{=}\PY{n}{datasample}\PY{p}{(}\PY{l+m+mi}{1}\PY{p}{:}\PY{n}{N}\PY{p}{,}\PY{l+m+mi}{2}\PY{p}{,}\PY{l+s}{\PYZsq{}}\PY{l+s}{Replace\PYZsq{}}\PY{p}{,}\PY{n}{false}\PY{p}{)}\PY{p}{;}
        \PY{c}{\PYZpc{} Se utilzia la información (cromosomas) de cada padre}
        \PY{n}{padre\PYZus{}1} \PY{p}{=} \PY{n}{OG\PYZus{}Cromosomas\PYZus{}padres}\PY{p}{(}\PY{p}{:}\PY{p}{,}\PY{p}{(}\PY{n}{padres}\PY{p}{(}\PY{l+m+mi}{1}\PY{p}{)}\PY{o}{\PYZhy{}}\PY{l+m+mi}{1}\PY{p}{)}\PY{o}{*}\PY{n}{OG\PYZus{}cant\PYZus{}lotes}\PY{o}{+}\PY{l+m+mi}{1}\PY{p}{:}\PY{c}{...}
        \PY{p}{(}\PY{n}{padres}\PY{p}{(}\PY{l+m+mi}{1}\PY{p}{)}\PY{o}{\PYZhy{}}\PY{l+m+mi}{1}\PY{p}{)}\PY{o}{*}\PY{n}{OG\PYZus{}cant\PYZus{}lotes}\PY{o}{+}\PY{n}{OG\PYZus{}cant\PYZus{}lotes}\PY{p}{)}\PY{p}{;}
        \PY{n}{padre\PYZus{}2} \PY{p}{=} \PY{n}{OG\PYZus{}Cromosomas\PYZus{}padres}\PY{p}{(}\PY{p}{:}\PY{p}{,}\PY{p}{(}\PY{n}{padres}\PY{p}{(}\PY{l+m+mi}{2}\PY{p}{)}\PY{o}{\PYZhy{}}\PY{l+m+mi}{1}\PY{p}{)}\PY{o}{*}\PY{n}{OG\PYZus{}cant\PYZus{}lotes}\PY{o}{+}\PY{l+m+mi}{1}\PY{p}{:}\PY{c}{...}
        \PY{p}{(}\PY{n}{padres}\PY{p}{(}\PY{l+m+mi}{2}\PY{p}{)}\PY{o}{\PYZhy{}}\PY{l+m+mi}{1}\PY{p}{)}\PY{o}{*}\PY{n}{OG\PYZus{}cant\PYZus{}lotes}\PY{o}{+}\PY{n}{OG\PYZus{}cant\PYZus{}lotes}\PY{p}{)}\PY{p}{;}
        \PY{c}{\PYZpc{} Se determina un punto de cruce, el cual es aleatorio entre 1 y la}
        \PY{c}{\PYZpc{} cantidad de lotes \PYZhy{}1}
        \PY{n}{punto\PYZus{}cruce}\PY{p}{=}\PY{n}{min}\PY{p}{(}\PY{n}{randi}\PY{p}{(}\PY{p}{[}\PY{l+m+mi}{1} \PY{n}{OG\PYZus{}cant\PYZus{}lotes}\PY{p}{]}\PY{p}{,}\PY{l+m+mi}{1}\PY{p}{,}\PY{l+m+mi}{1}\PY{p}{)}\PY{p}{,}\PY{n}{OG\PYZus{}cant\PYZus{}lotes}\PY{o}{\PYZhy{}}\PY{l+m+mi}{1}\PY{p}{)}\PY{p}{;}
        \PY{c}{\PYZpc{} Se construyen los hijos con la información de los padres teniendo en}
        \PY{c}{\PYZpc{} cuenta el punto de cruce}
        \PY{n}{hijo\PYZus{}1}\PY{p}{=}\PY{n+nb}{cat}\PY{p}{(}\PY{l+m+mi}{2}\PY{p}{,}\PY{n}{padre\PYZus{}1}\PY{p}{(}\PY{p}{:}\PY{p}{,}\PY{l+m+mi}{1}\PY{p}{:}\PY{n}{punto\PYZus{}cruce}\PY{p}{)}\PY{p}{,}\PY{n}{padre\PYZus{}2}\PY{p}{(}\PY{p}{:}\PY{p}{,}\PY{n}{punto\PYZus{}cruce}\PY{o}{+}\PY{l+m+mi}{1}\PY{p}{:}\PY{n}{OG\PYZus{}cant\PYZus{}lotes}\PY{p}{)}\PY{p}{)}\PY{p}{;}
        \PY{n}{hijo\PYZus{}2}\PY{p}{=}\PY{n+nb}{cat}\PY{p}{(}\PY{l+m+mi}{2}\PY{p}{,}\PY{n}{padre\PYZus{}2}\PY{p}{(}\PY{p}{:}\PY{p}{,}\PY{l+m+mi}{1}\PY{p}{:}\PY{n}{punto\PYZus{}cruce}\PY{p}{)}\PY{p}{,}\PY{n}{padre\PYZus{}1}\PY{p}{(}\PY{p}{:}\PY{p}{,}\PY{n}{punto\PYZus{}cruce}\PY{o}{+}\PY{l+m+mi}{1}\PY{p}{:}\PY{n}{OG\PYZus{}cant\PYZus{}lotes}\PY{p}{)}\PY{p}{)}\PY{p}{;}
        \PY{c}{\PYZpc{} Se declara el vector donde se almacena la función objetivo del hijo 1, y}
        \PY{c}{\PYZpc{} se evalua utilziando la función \PYZsq{}Evaluar\PYZus{}individuos\PYZsq{}}
        \PY{n}{objetivo\PYZus{}1}\PY{p}{=}\PY{n+nb}{zeros}\PY{p}{(}\PY{n}{OG\PYZus{}cant\PYZus{}objetivos}\PY{o}{+}\PY{l+m+mi}{1}\PY{p}{,}\PY{l+m+mi}{1}\PY{p}{)}\PY{p}{;}
        
        \PY{p}{[}\PY{n}{hijo\PYZus{}1}\PY{p}{,} \PY{n}{objetivo\PYZus{}1}\PY{p}{]}\PY{p}{=}\PY{n}{Evaluar\PYZus{}individuos}\PY{p}{(}\PY{n}{hijo\PYZus{}1}\PY{p}{,}\PY{n}{objetivo\PYZus{}1} \PY{p}{,} \PY{l+m+mi}{1}\PY{p}{,}  \PY{c}{...}
        \PY{n}{OG\PYZus{}cant\PYZus{}productos}\PY{p}{,} \PY{n}{OG\PYZus{}cant\PYZus{}lotes}\PY{p}{,} \PY{n}{PMS}\PY{p}{,} \PY{n}{OG\PYZus{}Covkkp}\PY{p}{,} \PY{c}{...}
        \PY{n}{OG\PYZus{}precio\PYZus{}venta}\PY{p}{,}\PY{n}{OG\PYZus{}rendimiento}\PY{p}{,} \PY{n}{OG\PYZus{}areas}\PY{p}{,} \PY{c}{...}
        \PY{n}{OG\PYZus{}demanda}\PY{p}{,}\PY{n}{OG\PYZus{}familia\PYZus{}botanica}\PY{p}{,}\PY{n}{OG\PYZus{}familia\PYZus{}venta}\PY{p}{)}\PY{p}{;}
        
        \PY{c}{\PYZpc{} Se declara el vector donde se almacena la función objetivo del hijo 2, y}
        \PY{c}{\PYZpc{} se evalua utilziando la función \PYZsq{}Evaluar\PYZus{}individuos\PYZsq{}}
        \PY{n}{objetivo\PYZus{}2}\PY{p}{=}\PY{n+nb}{zeros}\PY{p}{(}\PY{n}{OG\PYZus{}cant\PYZus{}objetivos}\PY{o}{+}\PY{l+m+mi}{1}\PY{p}{,}\PY{l+m+mi}{1}\PY{p}{)}\PY{p}{;}
        
        \PY{p}{[}\PY{n}{hijo\PYZus{}2}\PY{p}{,} \PY{n}{objetivo\PYZus{}2}\PY{p}{]}\PY{p}{=}\PY{n}{Evaluar\PYZus{}individuos}\PY{p}{(}\PY{n}{hijo\PYZus{}2}\PY{p}{,}\PY{n}{objetivo\PYZus{}2} \PY{p}{,} \PY{l+m+mi}{1}\PY{p}{,}  \PY{c}{...}
        \PY{n}{OG\PYZus{}cant\PYZus{}productos}\PY{p}{,} \PY{n}{OG\PYZus{}cant\PYZus{}lotes}\PY{p}{,} \PY{n}{PMS}\PY{p}{,} \PY{n}{OG\PYZus{}Covkkp}\PY{p}{,} \PY{c}{...}
        \PY{n}{OG\PYZus{}precio\PYZus{}venta}\PY{p}{,}\PY{n}{OG\PYZus{}rendimiento}\PY{p}{,} \PY{n}{OG\PYZus{}areas}\PY{p}{,} \PY{c}{...}
        \PY{n}{OG\PYZus{}demanda}\PY{p}{,}\PY{n}{OG\PYZus{}familia\PYZus{}botanica}\PY{p}{,}\PY{n}{OG\PYZus{}familia\PYZus{}venta}\PY{p}{)}\PY{p}{;}
        
        \PY{c}{\PYZpc{} Se concatena (almacena) la unformación de ambos hijos en forma de}
        \PY{c}{\PYZpc{} cromosoma y su respectivo valor de función de ajuste}
        \PY{c}{\PYZpc{} Cromosoma}
        \PY{n}{hijos}\PY{p}{=}\PY{n+nb}{cat}\PY{p}{(}\PY{l+m+mi}{2}\PY{p}{,}\PY{n}{hijo\PYZus{}1}\PY{p}{,}\PY{n}{hijo\PYZus{}2}\PY{p}{)}\PY{p}{;}
        \PY{c}{\PYZpc{} Valor de la función de ajuste}
        \PY{n}{objetivos}\PY{p}{=}\PY{n+nb}{cat}\PY{p}{(}\PY{l+m+mi}{2}\PY{p}{,}\PY{n}{objetivo\PYZus{}1}\PY{p}{,}\PY{n}{objetivo\PYZus{}2}\PY{p}{)}\PY{p}{;}
        \PY{k}{else}
        \PY{c}{\PYZpc{} Se declaran dos hijos vacios, para el caso de mutación}
        \PY{n}{hijo\PYZus{}1} \PY{p}{=} \PY{p}{[}\PY{p}{]}\PY{p}{;}
        \PY{n}{hijo\PYZus{}2} \PY{p}{=} \PY{p}{[}\PY{p}{]}\PY{p}{;}
        \PY{c}{\PYZpc{} Se seleccionan al azar dos padres}
        \PY{n}{padres}\PY{p}{=}\PY{n}{datasample}\PY{p}{(}\PY{l+m+mi}{1}\PY{p}{:}\PY{n}{N}\PY{p}{,}\PY{l+m+mi}{2}\PY{p}{,}\PY{l+s}{\PYZsq{}}\PY{l+s}{Replace\PYZsq{}}\PY{p}{,}\PY{n}{false}\PY{p}{)}\PY{p}{;}
        \PY{c}{\PYZpc{} Se utiliza la información (cromosomas) del padre 1}
        \PY{n}{padre\PYZus{}1} \PY{p}{=} \PY{n}{OG\PYZus{}Cromosomas\PYZus{}padres}\PY{p}{(}\PY{p}{:}\PY{p}{,}\PY{p}{(}\PY{n}{padres}\PY{p}{(}\PY{l+m+mi}{1}\PY{p}{)}\PY{o}{\PYZhy{}}\PY{l+m+mi}{1}\PY{p}{)}\PY{o}{*}\PY{n}{OG\PYZus{}cant\PYZus{}lotes}\PY{o}{+}\PY{l+m+mi}{1}\PY{p}{:}\PY{c}{...}
        \PY{p}{(}\PY{n}{padres}\PY{p}{(}\PY{l+m+mi}{1}\PY{p}{)}\PY{o}{\PYZhy{}}\PY{l+m+mi}{1}\PY{p}{)}\PY{o}{*}\PY{n}{OG\PYZus{}cant\PYZus{}lotes}\PY{o}{+}\PY{n}{OG\PYZus{}cant\PYZus{}lotes}\PY{p}{)}\PY{p}{;}
        \PY{c}{\PYZpc{} Se iguala la información del padre a la del hijo}
        \PY{n}{hijo\PYZus{}1}\PY{p}{=}\PY{n}{padre\PYZus{}1}\PY{p}{;}
        \PY{c}{\PYZpc{} Se identifica el punto (subcromosoma) donde se realizará la}
        \PY{c}{\PYZpc{} mutación}
        \PY{n}{punto\PYZus{}mutacion\PYZus{}1}\PY{p}{=}\PY{n}{randi}\PY{p}{(}\PY{p}{[}\PY{l+m+mi}{1} \PY{n}{OG\PYZus{}cant\PYZus{}lotes}\PY{p}{]}\PY{p}{,}\PY{l+m+mi}{1}\PY{p}{,}\PY{l+m+mi}{1}\PY{p}{)}\PY{p}{;}
        \PY{c}{\PYZpc{} un único individuo y un único lote}
        
        \PY{c}{\PYZpc{} Se genera la mutación utilizando una función}
        \PY{n}{mutacion} \PY{p}{=} \PY{n}{crear\PYZus{}mutacion}\PY{p}{(} \PY{l+m+mi}{1}\PY{p}{,} \PY{n}{OG\PYZus{}cant\PYZus{}objetivos}\PY{p}{,} \PY{n}{OG\PYZus{}cant\PYZus{}periodos}\PY{p}{,} \PY{l+m+mi}{1}\PY{p}{)}\PY{p}{;}
        
        
        \PY{c}{\PYZpc{} Se actualiza el cromosoma en la solución hijo}
        \PY{n}{hijo\PYZus{}1}\PY{p}{(}\PY{p}{:}\PY{p}{,}\PY{n}{punto\PYZus{}mutacion\PYZus{}1}\PY{p}{)}\PY{p}{=}\PY{n}{mutacion}\PY{p}{;}
        \PY{c}{\PYZpc{} Se crea una variable para almacenar la solución del hijo 1}
        \PY{n}{objetivo\PYZus{}1}\PY{p}{=}\PY{n+nb}{zeros}\PY{p}{(}\PY{n}{OG\PYZus{}cant\PYZus{}objetivos}\PY{o}{+}\PY{l+m+mi}{1}\PY{p}{,}\PY{l+m+mi}{1}\PY{p}{)}\PY{p}{;}
        
        \PY{c}{\PYZpc{} Se evalua la solución del hijo 1}
        
        \PY{p}{[}\PY{n}{hijo\PYZus{}1}\PY{p}{,} \PY{n}{objetivo\PYZus{}1}\PY{p}{]}\PY{p}{=}\PY{n}{Evaluar\PYZus{}individuos}\PY{p}{(}\PY{n}{hijo\PYZus{}1}\PY{p}{,}\PY{n}{objetivo\PYZus{}1} \PY{p}{,} \PY{l+m+mi}{1}\PY{p}{,}  \PY{c}{...}
        \PY{n}{OG\PYZus{}cant\PYZus{}productos}\PY{p}{,} \PY{n}{OG\PYZus{}cant\PYZus{}lotes}\PY{p}{,} \PY{n}{PMS}\PY{p}{,} \PY{n}{OG\PYZus{}Covkkp}\PY{p}{,} \PY{c}{...}
        \PY{n}{OG\PYZus{}precio\PYZus{}venta}\PY{p}{,}\PY{n}{OG\PYZus{}rendimiento}\PY{p}{,} \PY{n}{OG\PYZus{}areas}\PY{p}{,} \PY{n}{OG\PYZus{}demanda}\PY{p}{,}\PY{c}{...}
        \PY{n}{OG\PYZus{}familia\PYZus{}botanica}\PY{p}{,}\PY{n}{OG\PYZus{}familia\PYZus{}venta}\PY{p}{)}\PY{p}{;}
        
        
        \PY{c}{\PYZpc{} Se utiliza la información (cromosomas) del padre 2}
        \PY{n}{padre\PYZus{}2} \PY{p}{=} \PY{n}{OG\PYZus{}Cromosomas\PYZus{}padres}\PY{p}{(}\PY{p}{:}\PY{p}{,}\PY{p}{(}\PY{n}{padres}\PY{p}{(}\PY{l+m+mi}{2}\PY{p}{)}\PY{o}{\PYZhy{}}\PY{l+m+mi}{1}\PY{p}{)}\PY{o}{*}\PY{n}{OG\PYZus{}cant\PYZus{}lotes}\PY{o}{+}\PY{l+m+mi}{1}\PY{p}{:}\PY{c}{...}
        \PY{p}{(}\PY{n}{padres}\PY{p}{(}\PY{l+m+mi}{2}\PY{p}{)}\PY{o}{\PYZhy{}}\PY{l+m+mi}{1}\PY{p}{)}\PY{o}{*}\PY{n}{OG\PYZus{}cant\PYZus{}lotes}\PY{o}{+}\PY{n}{OG\PYZus{}cant\PYZus{}lotes}\PY{p}{)}\PY{p}{;}
        \PY{c}{\PYZpc{} Se iguala la información del padre a la del hijo}
        \PY{n}{hijo\PYZus{}2}\PY{p}{=}\PY{n}{padre\PYZus{}2}\PY{p}{;}
        \PY{c}{\PYZpc{} Se identifica el punto (subcromosoma) donde se realizará la}
        \PY{c}{\PYZpc{} mutación}
        \PY{n}{punto\PYZus{}mutacion\PYZus{}2}\PY{p}{=}\PY{n}{randi}\PY{p}{(}\PY{p}{[}\PY{l+m+mi}{1} \PY{n}{OG\PYZus{}cant\PYZus{}lotes}\PY{p}{]}\PY{p}{,}\PY{l+m+mi}{1}\PY{p}{,}\PY{l+m+mi}{1}\PY{p}{)}\PY{p}{;}
        
        \PY{c}{\PYZpc{} Se genera la mutación utilizando una función}
        
        \PY{n}{mutacion} \PY{p}{=} \PY{n}{crear\PYZus{}mutacion}\PY{p}{(} \PY{l+m+mi}{1}\PY{p}{,} \PY{n}{OG\PYZus{}cant\PYZus{}objetivos}\PY{p}{,} \PY{n}{OG\PYZus{}cant\PYZus{}periodos}\PY{p}{,} \PY{l+m+mi}{1}\PY{p}{)}\PY{p}{;}
        
        \PY{c}{\PYZpc{} Se actualiza el cromosoma en la solución hijo}
        \PY{n}{hijo\PYZus{}2}\PY{p}{(}\PY{p}{:}\PY{p}{,}\PY{n}{punto\PYZus{}mutacion\PYZus{}2}\PY{p}{)}\PY{p}{=}\PY{n}{mutacion}\PY{p}{;}
        \PY{c}{\PYZpc{}         Se crea una variable para almacenar la solución del hijo 2}
        \PY{n}{objetivo\PYZus{}2}\PY{p}{=}\PY{n+nb}{zeros}\PY{p}{(}\PY{n}{OG\PYZus{}cant\PYZus{}objetivos}\PY{o}{+}\PY{l+m+mi}{1}\PY{p}{,}\PY{l+m+mi}{1}\PY{p}{)}\PY{p}{;}
        
        \PY{c}{\PYZpc{} Se evalua la solución del hijo 2}
        
        \PY{p}{[}\PY{n}{hijo\PYZus{}2}\PY{p}{,} \PY{n}{objetivo\PYZus{}2}\PY{p}{]}\PY{p}{=}\PY{n}{Evaluar\PYZus{}individuos}\PY{p}{(}\PY{n}{hijo\PYZus{}2}\PY{p}{,}\PY{n}{objetivo\PYZus{}2} \PY{p}{,} \PY{l+m+mi}{1}\PY{p}{,}  \PY{c}{...}
        \PY{n}{OG\PYZus{}cant\PYZus{}productos}\PY{p}{,} \PY{n}{OG\PYZus{}cant\PYZus{}lotes}\PY{p}{,} \PY{n}{PMS}\PY{p}{,} \PY{n}{OG\PYZus{}Covkkp}\PY{p}{,} \PY{n}{OG\PYZus{}precio\PYZus{}venta}\PY{p}{,}\PY{n}{OG\PYZus{}rendimiento}\PY{p}{,} \PY{n}{OG\PYZus{}areas}\PY{p}{,} \PY{c}{...}
        \PY{n}{OG\PYZus{}demanda}\PY{p}{,}\PY{n}{OG\PYZus{}familia\PYZus{}botanica}\PY{p}{,}\PY{n}{OG\PYZus{}familia\PYZus{}venta}\PY{p}{)}\PY{p}{;}
        
        \PY{c}{\PYZpc{} Se concatena (almacena) la unformación de ambos hijos en forma de}
        \PY{c}{\PYZpc{} cromosoma y su respectivo valor de función de ajuste}
        \PY{c}{\PYZpc{} Cromosoma}
        \PY{n}{hijos}\PY{p}{=}\PY{n+nb}{cat}\PY{p}{(}\PY{l+m+mi}{2}\PY{p}{,}\PY{n}{hijo\PYZus{}1}\PY{p}{,}\PY{n}{hijo\PYZus{}2}\PY{p}{)}\PY{p}{;}
        \PY{c}{\PYZpc{} Valor de la función de ajuste}
        \PY{n}{objetivos}\PY{p}{=}\PY{n+nb}{cat}\PY{p}{(}\PY{l+m+mi}{2}\PY{p}{,}\PY{n}{objetivo\PYZus{}1}\PY{p}{,}\PY{n}{objetivo\PYZus{}2}\PY{p}{)}\PY{p}{;}
        \PY{k}{end}
        \PY{c}{\PYZpc{} Se actualiza la información de la solución}
        \PY{c}{\PYZpc{} solución}
        \PY{n}{Cromosomas\PYZus{}hijos}\PY{p}{(}\PY{p}{:}\PY{p}{,}\PY{p}{(}\PY{n+nb}{i}\PY{o}{\PYZhy{}}\PY{l+m+mi}{1}\PY{p}{)}\PY{o}{*}\PY{n}{OG\PYZus{}cant\PYZus{}lotes}\PY{o}{*}\PY{l+m+mi}{2}\PY{o}{+}\PY{l+m+mi}{1}\PY{p}{:}\PY{p}{(}\PY{n+nb}{i}\PY{o}{\PYZhy{}}\PY{l+m+mi}{1}\PY{p}{)}\PY{o}{*}\PY{n}{OG\PYZus{}cant\PYZus{}lotes}\PY{o}{*}\PY{l+m+mi}{2}\PY{o}{+}\PY{n}{OG\PYZus{}cant\PYZus{}lotes}\PY{o}{*}\PY{l+m+mi}{2}\PY{p}{)}\PY{p}{=}\PY{n}{hijos}\PY{p}{;}
        \PY{c}{\PYZpc{}Valor de las funciones de ajuste}
        \PY{n}{objetivos\PYZus{}hijos}\PY{p}{(}\PY{p}{:}\PY{p}{,}\PY{p}{(}\PY{n+nb}{i}\PY{o}{\PYZhy{}}\PY{l+m+mi}{1}\PY{p}{)}\PY{o}{*}\PY{l+m+mi}{2}\PY{o}{+}\PY{l+m+mi}{1}\PY{p}{:}\PY{p}{(}\PY{n+nb}{i}\PY{o}{\PYZhy{}}\PY{l+m+mi}{1}\PY{p}{)}\PY{o}{*}\PY{l+m+mi}{2}\PY{o}{+}\PY{l+m+mi}{2}\PY{p}{)} \PY{p}{=}\PY{n}{objetivos}\PY{p}{;}
        \PY{k}{end}
        \PY{k}{end}
\end{Verbatim}


    \hypertarget{algortimo-para-crear-la-mutaciuxf3n-crear_mutacion.m}{%
\subsection{Algortimo para crear la mutación
(crear\_mutacion.m)}\label{algortimo-para-crear-la-mutaciuxf3n-crear_mutacion.m}}

Se contruye una función denominada `crear-mutacion', la cual en escencia
está estructurada como la función \% -inicializar-cromosomas' pero
enfocada en un único lote. Los parámetros de entrada son CM\_poblacion
(población, la cual en este caso es un único individuo),
CM\_cant\_objetivos (cantidad de objetivos a evaluar),
CM\_cant\_periodos (la cantidad de periodos del horizonte de planeación)
y CM\_cant\_lotes (cantidad de lotes a crear, en este caso, un único
lote). Como salida la función presenta un único cromosoma denominado:
`CM\_Cromosoma'.

    \begin{Verbatim}[commandchars=\\\{\}]
{\color{incolor}In [{\color{incolor} }]:} \PY{k}{function}\PY{err}{ }\PY{err}{[} \PY{n+nf}{CM\PYZus{}Cromosoma} \PY{p}{]} \PY{p}{=} \PY{n}{crear\PYZus{}mutacion}\PY{p}{(} \PY{n}{CM\PYZus{}poblacion}\PY{p}{,}\PY{c}{...}
        \PY{n}{CM\PYZus{}cant\PYZus{}objetivos}\PY{p}{,} \PY{n}{CM\PYZus{}cant\PYZus{}periodos}\PY{p}{,} \PY{n}{CM\PYZus{}cant\PYZus{}lotes}\PY{p}{)}
        \PY{c}{\PYZpc{}Se cargan los parámetros}
        \PY{n}{load}\PY{p}{(}\PY{l+s}{\PYZsq{}}\PY{l+s}{parametros\PYZus{}maduracion.mat\PYZsq{}}\PY{p}{)}
        \PY{c+cm}{\PYZpc{}\PYZob{}}
        \PY{c+cm}{PPP = Productos Por Periodo}
        \PY{c+cm}{MTS = Matriz Tiempos Siembra}
        \PY{c+cm}{MFS = Matriz Fecha Siembra}
        \PY{c+cm}{PM  = Periodo de maduración}
        \PY{c+cm}{PMS = Periodo de maduración en semanas}
        \PY{c+cm}{para la prueba trabajaré con 5 sujetos}
        \PY{c+cm}{Cálculo de variables y parámetros}
        \PY{c+cm}{Duración del proyecto}
        \PY{c+cm}{\PYZpc{}\PYZcb{}}
        \PY{n}{T}\PY{p}{=}\PY{n}{CM\PYZus{}cant\PYZus{}periodos}\PY{p}{;}
        \PY{c}{\PYZpc{} Cantidad de Lotes}
        \PY{n}{L}\PY{p}{=}\PY{n}{CM\PYZus{}cant\PYZus{}lotes}\PY{p}{;}
        \PY{c}{\PYZpc{} Cantidad de cromosomas}
        \PY{n}{C}\PY{p}{=}\PY{n}{CM\PYZus{}poblacion}\PY{p}{;}
        \PY{c}{\PYZpc{} Contador para los bucles}
        \PY{n}{CONTADOR}\PY{p}{=}\PY{l+m+mi}{0}\PY{p}{;}
        \PY{c}{\PYZpc{} Se contruye la matriz donde se almacenarán la función de ajuste, seguida}
        \PY{c}{\PYZpc{} de las funciones objetivo}
        \PY{n}{Matriz\PYZus{}objetivos}\PY{p}{=}\PY{n+nb}{zeros}\PY{p}{(}\PY{n}{CM\PYZus{}cant\PYZus{}objetivos}\PY{o}{+}\PY{l+m+mi}{1}\PY{p}{,}\PY{n}{CM\PYZus{}poblacion}\PY{p}{)}\PY{p}{;}
        \PY{c}{\PYZpc{}\PYZpc{}}
        \PY{c}{\PYZpc{} Creación del cromosoma}
        \PY{n}{CM\PYZus{}Cromosoma}\PY{p}{=}\PY{n+nb}{zeros}\PY{p}{(}\PY{n}{T}\PY{p}{,}\PY{n}{L}\PY{o}{*}\PY{n}{C}\PY{p}{)}\PY{p}{;}
        \PY{c}{\PYZpc{} Contador de periodos del proyecto}
        \PY{n}{CPP}\PY{p}{=}\PY{l+m+mi}{1}\PY{p}{:}\PY{l+m+mi}{1}\PY{p}{:}\PY{n}{T}\PY{p}{;}
        \PY{c+cm}{\PYZpc{}\PYZob{}}
        \PY{c+cm}{Bucle para asignar aleatoriamente los productos durante el horizonte de}
        \PY{c+cm}{planeación}
        \PY{c+cm}{Cromosomas auxiliares para llevar la asignación de tiempos y bloqueo de}
        \PY{c+cm}{lotes}
        \PY{c+cm}{Ayuda a organizar los productos en los diversos periodos}
        \PY{c+cm}{\PYZpc{}\PYZcb{}}
        \PY{n}{Cromosoma\PYZus{}aux1}\PY{p}{=}\PY{l+m+mi}{0}\PY{o}{*}\PY{n}{CM\PYZus{}Cromosoma}\PY{p}{;}
        \PY{c}{\PYZpc{} Almacena el contador de meses en que dura ocupado cada terreno}
        \PY{n}{Cromosoma\PYZus{}aux2}\PY{p}{=}\PY{n}{Cromosoma\PYZus{}aux1}\PY{p}{;}
        \PY{c}{\PYZpc{} Ayuda a verificar condiciones del cromosoma}
        \PY{n}{Cromosoma\PYZus{}aux3}\PY{p}{=}\PY{n}{Cromosoma\PYZus{}aux1}\PY{o}{+}\PY{l+m+mi}{1}\PY{p}{;}
        \PY{c}{\PYZpc{} se realiza un bucle durante la duración del proyecto, por ahorro en}
        \PY{c}{\PYZpc{} recursos computacionales, no se tiene en cuenta los tres últimos periodos}
        \PY{c}{\PYZpc{} ya que ningún producto puede ser sembrado en ese instante:}
        \PY{k}{for} \PY{n}{tiempo}\PY{p}{=}\PY{l+m+mi}{1}\PY{p}{:}\PY{l+m+mi}{1}\PY{p}{:}\PY{n}{T}\PY{o}{\PYZhy{}}\PY{l+m+mi}{3}
        \PY{c}{\PYZpc{}Para cada periodo se asignan aletoriamente los 25 productos en L}
        \PY{c}{\PYZpc{}lotes, siempre y cuando estos puedan ser sembrados en ese instante y}
        \PY{c}{\PYZpc{}existan lotes disponibles}
        \PY{n}{CM\PYZus{}Cromosoma}\PY{p}{(}\PY{n}{tiempo}\PY{p}{,}\PY{p}{:}\PY{p}{)}\PY{p}{=} \PY{p}{(}\PY{n}{datasample}\PY{p}{(}\PY{n+nb}{find}\PY{p}{(}\PY{n}{MFS}\PY{p}{(}\PY{p}{:}\PY{p}{,}\PY{n}{tiempo}\PY{p}{)}\PY{o}{\PYZti{}=}\PY{l+m+mi}{0}\PY{p}{)}\PY{p}{,}\PY{n}{L}\PY{o}{*}\PY{n}{C}\PY{p}{)}\PY{o}{\PYZsq{}}\PY{p}{)}\PY{c}{....}
        \PY{o}{*}\PY{n}{Cromosoma\PYZus{}aux3}\PY{p}{(}\PY{n}{tiempo}\PY{p}{,}\PY{p}{:}\PY{p}{)}\PY{p}{;}
        \PY{n}{Cromosoma\PYZus{}aux1}\PY{p}{(}\PY{n}{tiempo}\PY{p}{,}\PY{p}{:}\PY{p}{)}\PY{p}{=}\PY{n}{CM\PYZus{}Cromosoma}\PY{p}{(}\PY{n}{tiempo}\PY{p}{,}\PY{p}{:}\PY{p}{)}\PY{p}{;}
        \PY{c}{\PYZpc{}Determino un listado de los productos que pueden ser sembrados en cada}
        \PY{c}{\PYZpc{}periodo}
        \PY{n}{var\PYZus{}aux1}\PY{p}{=}\PY{n+nb}{find}\PY{p}{(}\PY{n}{MFS}\PY{p}{(}\PY{p}{:}\PY{p}{,}\PY{n}{tiempo}\PY{p}{)}\PY{o}{\PYZti{}=}\PY{l+m+mi}{0}\PY{p}{)}\PY{p}{;}
        \PY{k}{for} \PY{n}{listado}\PY{p}{=}\PY{l+m+mi}{1}\PY{p}{:}\PY{n+nb}{length}\PY{p}{(}\PY{n}{var\PYZus{}aux1}\PY{p}{)}
        \PY{n}{Cromosoma\PYZus{}aux2}\PY{p}{(}\PY{n}{tiempo}\PY{p}{,}\PY{p}{:}\PY{p}{)} \PY{p}{=} \PY{n}{Cromosoma\PYZus{}aux2}\PY{p}{(}\PY{n}{tiempo}\PY{p}{,}\PY{p}{:}\PY{p}{)} \PY{o}{+} \PY{p}{(}\PY{n}{CM\PYZus{}Cromosoma}\PY{p}{(}\PY{n}{tiempo}\PY{p}{,}\PY{p}{:}\PY{p}{)}\PY{c}{...}
        \PY{o}{==}\PY{n}{var\PYZus{}aux1}\PY{p}{(}\PY{n}{listado}\PY{p}{)}\PY{p}{)}\PY{o}{*}\PY{n}{PM}\PY{p}{(}\PY{n}{var\PYZus{}aux1}\PY{p}{(}\PY{n}{listado}\PY{p}{)}\PY{p}{)}\PY{p}{;}
        \PY{k}{end}
        \PY{k}{while} \PY{n}{tiempo} \PY{o}{\PYZgt{}} \PY{l+m+mi}{1}
        \PY{n}{Cromosoma\PYZus{}aux2}\PY{p}{(}\PY{n}{tiempo}\PY{p}{,}\PY{p}{:}\PY{p}{)} \PY{p}{=} \PY{n}{Cromosoma\PYZus{}aux2}\PY{p}{(}\PY{n}{tiempo}\PY{o}{\PYZhy{}}\PY{l+m+mi}{1}\PY{p}{,}\PY{p}{:}\PY{p}{)} \PY{o}{\PYZhy{}} \PY{l+m+mi}{1}\PY{p}{;}
        \PY{n}{CM\PYZus{}Cromosoma}\PY{p}{(}\PY{n}{tiempo}\PY{p}{,}\PY{p}{:}\PY{p}{)}\PY{p}{=}\PY{p}{(}\PY{n}{datasample}\PY{p}{(}\PY{n+nb}{find}\PY{p}{(}\PY{n}{MFS}\PY{p}{(}\PY{p}{:}\PY{p}{,}\PY{n}{tiempo}\PY{p}{)}\PY{o}{\PYZti{}=}\PY{l+m+mi}{0}\PY{p}{)}\PY{p}{,}\PY{n}{L}\PY{o}{*}\PY{n}{C}\PY{p}{)}\PY{o}{\PYZsq{}}\PY{p}{)}\PY{o}{.*}\PY{c}{...}
        \PY{p}{(}\PY{n}{Cromosoma\PYZus{}aux2}\PY{p}{(}\PY{n}{tiempo}\PY{p}{,}\PY{p}{:}\PY{p}{)} \PY{o}{\PYZlt{}}\PY{l+m+mi}{1}\PY{p}{)}\PY{p}{;}
        \PY{n}{Cromosoma\PYZus{}aux1}\PY{p}{(}\PY{n}{tiempo}\PY{p}{,}\PY{p}{:}\PY{p}{)}\PY{p}{=}\PY{n}{CM\PYZus{}Cromosoma}\PY{p}{(}\PY{n}{tiempo}\PY{p}{,}\PY{p}{:}\PY{p}{)}\PY{p}{;}
        \PY{c}{\PYZpc{} Determino un listado de los productos que pueden ser sembrados en cada}
        \PY{c}{\PYZpc{} periodo}
        \PY{n}{var\PYZus{}aux1}\PY{p}{=}\PY{n+nb}{find}\PY{p}{(}\PY{n}{MFS}\PY{p}{(}\PY{p}{:}\PY{p}{,}\PY{n}{tiempo}\PY{p}{)}\PY{o}{\PYZti{}=}\PY{l+m+mi}{0}\PY{p}{)}\PY{p}{;}
        \PY{k}{for} \PY{n}{listado}\PY{p}{=}\PY{l+m+mi}{1}\PY{p}{:}\PY{n+nb}{length}\PY{p}{(}\PY{n}{var\PYZus{}aux1}\PY{p}{)}
        \PY{n}{Cromosoma\PYZus{}aux2}\PY{p}{(}\PY{n}{tiempo}\PY{p}{,}\PY{p}{:}\PY{p}{)} \PY{p}{=} \PY{n}{Cromosoma\PYZus{}aux2}\PY{p}{(}\PY{n}{tiempo}\PY{p}{,}\PY{p}{:}\PY{p}{)} \PY{o}{+} \PY{p}{(}\PY{n}{CM\PYZus{}Cromosoma}\PY{p}{(}\PY{n}{tiempo}\PY{p}{,}\PY{p}{:}\PY{p}{)}\PY{o}{==}\PY{c}{...}
        \PY{n}{var\PYZus{}aux1}\PY{p}{(}\PY{n}{listado}\PY{p}{)}\PY{p}{)}\PY{o}{*}\PY{n}{PM}\PY{p}{(}\PY{n}{var\PYZus{}aux1}\PY{p}{(}\PY{n}{listado}\PY{p}{)}\PY{p}{)}\PY{p}{;}
        \PY{k}{end}
        \PY{k}{break}
        \PY{k}{end}
        \PY{k}{end}
        \PY{k}{end}
\end{Verbatim}



    % Add a bibliography block to the postdoc
    
    
    
    \end{document}
